\textbf{Notes}:

\begin{itemize}
    \item Instructions accept a conditional test suffix, unless indicated via a \(\square\) symbol.
    \item Mnemonics support overloading.
    That is, the same mnemonic can have many argument signatures.
    Optional arguments are listed using square brackets \texttt{[optional]} versus mandatory arguments \texttt{<mandatory>}.
    \item For all arithmetic and logical instructions with signatures \texttt{<reg> <reg> <value>}, the first register is optional.
    If omitted, the supplied register is duplicated.
    I.e., \texttt{\$r, \$v} becomes \texttt{\$r, \$r, \$v}.
    \item All arithmetic operations (except \texttt{mod}) and the compare operation take a datatype.
\end{itemize}

\subsection{Data Transfer}

\subsubsection{Load}

\begin{lstlisting}[style=assembly]
    load <reg> <value>
\end{lstlisting}

Loads \texttt{<value>} as a word into register \texttt{<reg>}.
Note, \texttt{<value>} may only specify a 32-bit immediate, so the upper 32 bits will always be zeroed in this case.
To load a full 64-bit immediate, use \texttt{loadi}.

\begin{lstlisting}[style=rtn]
    Reg[reg] <- value
\end{lstlisting}

\subsubsection{Load Upper}

\begin{lstlisting}[style=assembly]
    loadu <reg> <value>
\end{lstlisting}

Loads \texttt{<value>} as a half-word into the upper half of register \texttt{<reg>}.

\begin{lstlisting}[style=rtn]
    Reg[reg][32:] <- value
\end{lstlisting}

\subsubsection{Load Immediate}

\textit{Pseudo-instruction}

\begin{lstlisting}[style=assembly]
    loadi <reg> <imm>
\end{lstlisting}

Expands to

\begin{lstlisting}[style=assembly]
    load <reg> <imm>[:32]
    loadu <reg> <imm>[32:]
\end{lstlisting}

Loads \texttt{<imm>} as a word into register \texttt{<reg>}.

\subsubsection{Zero}

\textit{Pseudo-instruction}

\begin{lstlisting}[style=assembly]
    zero <reg>
\end{lstlisting}

Expands to

\begin{lstlisting}[style=assembly]
    xor <reg> <reg>
\end{lstlisting}

Clears (zeroes) register \texttt{<reg>}.

\subsubsection{Store}

\begin{lstlisting}[style=assembly]
    store <reg> <addr>
\end{lstlisting}

Stores the contents of register \texttt{<reg>} at the given address \texttt{<addr>}.

\begin{lstlisting}[style=rtn]
    Mem[addr] <- Reg[reg]
\end{lstlisting}

\subsubsection{Convert}

\begin{lstlisting}[style=assembly]
    cvt<d1>2<d2> <reg> <reg>
\end{lstlisting}

Converts the second register from data-type \(d_1\) to \(d_2\) and store in the first register.

\begin{lstlisting}[style=rtn]
    Reg[reg1] <- cvt(Reg[reg2], d1, d2)
\end{lstlisting}

\subsection{Arithmetic}

\textbf{Note} that all mnemonics, except \texttt{mod}, expect a datatype flag.

\subsubsection{Addition}

\begin{lstlisting}[style=assembly]
    add <reg> <reg> <value>
\end{lstlisting}

Add the value in the second register to \texttt{<value>} and store in the first register.

\begin{lstlisting}[style=rtn]
    Reg[reg1] <- Reg[reg2] + value
\end{lstlisting}

\subsubsection{Subtraction}

\begin{lstlisting}[style=assembly]
    sub <reg> <reg> <value>
\end{lstlisting}

Subtract \texttt{<value>} from the value in the second register and store in the first register.

\begin{lstlisting}[style=rtn]
    Reg[reg1] <- Reg[reg2] - value
\end{lstlisting}

\subsubsection{Multiplication}

\begin{lstlisting}[style=assembly]
    mul <reg> <reg> <value>
\end{lstlisting}

Multiply the value in the second register by \texttt{<value>} and store in the first register.

\begin{lstlisting}[style=rtn,mathescape]
    Reg[reg1] <- Reg[reg2] $\times$ value
\end{lstlisting}

\subsubsection{Division}

\begin{lstlisting}[style=assembly]
    div <reg> <reg> <value>
\end{lstlisting}

Divide the value in the second register by \texttt{<value>} and store in the first register as a \textit{double}.

\begin{lstlisting}[style=rtn,mathescape]
    Reg[reg1] <- Reg[reg2] $\div$ value
\end{lstlisting}

\subsubsection{Modulo}

\begin{lstlisting}[style=assembly]
    mod <reg> <reg> <value>
\end{lstlisting}

Calculate the remainder when dividing the second register by the value.
The register is treated as a signed word; the value as a signed half-word.

\begin{lstlisting}[style=rtn,mathescape]
    Reg[reg1] <- Reg[reg2] $\mod$ value
\end{lstlisting}

\subsection{Branching}

\subsubsection{Compare}

\begin{lstlisting}[style=assembly]
    cmp <reg> <value>
\end{lstlisting}

Compare the value in the register by \texttt{<value>}; set the comparison bits in the \texttt{\$flag} register appropriately.
E.g., set \texttt{lt} iff \texttt{\$reg < \$value}.
\textbf{Note} the Z flag is set depending on value, not register.

\subsubsection{Branch}

\textit{Pseudo-instruction}

\begin{lstlisting}[style=assembly]
    b<cnd> <value>
\end{lstlisting}

Expands to

\begin{lstlisting}[style=assembly]
    load<cnd> $pc <value>
\end{lstlisting}

Branch to the given value if the flag's comparison bits match the conditional guard.

\subsubsection{Jump}

\textit{Pseudo-instruction}

\begin{lstlisting}[style=assembly]
    jmp <value>
\end{lstlisting}

Expands to

\begin{lstlisting}[style=assembly]
    load $pc <value>
\end{lstlisting}

An unconditional branch.

\subsection{Logical}

\subsubsection{And}

\begin{lstlisting}[style=assembly]
    and <reg> <reg> <value>
\end{lstlisting}

Computer the bitwise AND of the value in the second register and \texttt{<value>} and stores the result in the second register.

\begin{lstlisting}[style=rtn,mathescape]
    Reg[reg1] <- Reg[reg2] & value
\end{lstlisting}

\subsubsection{Not}

\begin{lstlisting}[style=assembly]
    not <reg> <reg>
\end{lstlisting}

Stores the inverse (bitwise NOT) of the value in the second register in the first register.

\begin{lstlisting}[style=rtn,mathescape]
    Reg[reg1] <- $\sim$ Reg[reg2]
\end{lstlisting}

\subsubsection{Or}

\begin{lstlisting}[style=assembly]
    and <reg> <reg> <value>
\end{lstlisting}

Computer the bitwise OR of the value in the second register and \texttt{<value>} and stores the result in the first register.

\begin{lstlisting}[style=rtn,mathescape]
    Reg[reg1] <- Reg[reg2] | value
\end{lstlisting}

\subsubsection{Exclusive-Or}

\begin{lstlisting}[style=assembly]
    xor <reg> <reg> <value>
\end{lstlisting}

Computer the bitwise exclusive OR of the value in the second register and \texttt{<value>} and stores the result in the first register.

\begin{lstlisting}[style=rtn,mathescape]
    Reg[reg1] <- Reg[reg2] ^ value
\end{lstlisting}

\subsubsection{Left Shift}

\begin{lstlisting}[style=assembly]
    shl <reg> <reg> <value>
\end{lstlisting}

Logically shift the value in the second register left by \texttt{<value>} and store the result in the first register.

\begin{lstlisting}[style=rtn,mathescape]
    Reg[reg1] <- Reg[reg2] $\ll$ value
\end{lstlisting}

\subsubsection{Right Shift}

\begin{lstlisting}[style=assembly]
    shl <reg> <reg> <value>
\end{lstlisting}

Logically shift the value in the second register right by \texttt{<value>} and store the result in the first register.

\begin{lstlisting}[style=rtn,mathescape]
    Reg[reg1] <- Reg[reg2] $\gg$ value
\end{lstlisting}

\subsubsection{Sign Extend}

\begin{lstlisting}[style=assembly]
    sext <reg> <value> <imm: 8>
\end{lstlisting}

Extend the signed data of \texttt{<imm>} bits and store as a word in the destination register.
That is, if the MSB is one, extend with ones, else extend with zeroes.
\textbf{All} bits not composing the data are set to one or zero.

\begin{lstlisting}[style=rtn,mathescape]
    if msb = 1 then
        Reg[reg] <- value | ($\sim$0 << imm)
    else
        Reg[reg] <- zero-extend(value, imm)
\end{lstlisting}

\subsubsection{Zero Extend}

\begin{lstlisting}[style=assembly]
    zext <reg> <value> <imm: 8>
\end{lstlisting}

Extend the unsigned data of \texttt{<imm>} bits and store as a word in the destination register.

\begin{lstlisting}[style=rtn,mathescape]
    Reg[reg] <- value & ((1 << imm) - 1)
\end{lstlisting}

\subsection{Subroutines}

\subsubsection{Jump and Link}

\begin{lstlisting}[style=assembly]
    jal [reg] <addr>
\end{lstlisting}

Performs a function call: stores the current \texttt{\$pc} in \texttt{<reg$_1$>} and jumps to \texttt{<addr>}.
By default, this register is \texttt{\$rpc}; the optional register argument changes this.

\begin{lstlisting}[style=rtn]
    Reg[reg1] <- Reg[$pc]
    Reg[$pc] <- addr
\end{lstlisting}

\subsubsection{Return}

\textit{Pseudo-instruction}

\begin{lstlisting}[style=assembly]
    ret [value]
\end{lstlisting}

Expands to

\begin{lstlisting}[style=assembly]
    load $ret <value>
    load $pc $rpc
\end{lstlisting}

If provided, loads the provided value into \$ret, and loads \$rpc into \$pc.

\subsection{Interrupts}

\subsubsection{Trigger Interrupt}

\textit{Pseudo-instruction}

\begin{lstlisting}[style=assembly]
    int <value>
\end{lstlisting}

Expands to

\begin{lstlisting}[style=assembly]
    or $isr <value>
\end{lstlisting}

Mask the \texttt{\$isr} register; trigger the given interrupt given the bit mask.

\subsubsection{Return from Interrupt}

\textit{Pseudo-instruction}

\begin{lstlisting}[style=assembly]
    rti
\end{lstlisting}

Expands to

\begin{lstlisting}[style=assembly]
    load $ipc $pc
    and $flag ~FLAG_INTERRUPT_BIT
\end{lstlisting}

Restores instruction pointer to pre-interrupt state, and unlocks future interrupts.

\subsection{Miscellaneous}

\subsubsection{No-Operation}

\begin{lstlisting}[style=assembly]
    nop
\end{lstlisting}

Do nothing; consume an instruction cycle.

\subsubsection{Exit}

\textit{Pseudo-instruction}

\begin{lstlisting}[style=assembly]
    exit [value]
\end{lstlisting}

Expands to

\begin{lstlisting}[style=assembly]
    load $ret <value>
    syscall <exit>
\end{lstlisting}

Exit or halt the given process, optionally with a provided exit code.

\subsubsection{System Call}

\begin{lstlisting}[style=assembly]
    syscall <value>
\end{lstlisting}

Invoke the system call mapped to the given value.
See the system call section for these mappings.

\begin{lstlisting}[style=rtn]
    pop stack frame
    Reg[$pc] <- old $pc
\end{lstlisting}
