\documentclass{article}
\usepackage[margin=0.75in]{geometry}
\usepackage{makecell}
\usepackage{longtable}
\usepackage{amsmath, amssymb}
\usepackage{listings}
\usepackage{xcolor}
\usepackage{tikz}
\usepackage{tcolorbox}
\usepackage{multirow}
\usepackage{float}

\lstdefinestyle{bashconsole}{
    backgroundcolor=\color{black!5},
    basicstyle=\ttfamily\color{black},
    keywordstyle=\color{blue}\bfseries,
    commentstyle=\color{green!70!black},
    stringstyle=\color{red!70!black},
    numberstyle=\tiny\color{gray},
    breaklines=true,
    frame=single,
    rulecolor=\color{black!20},
    xleftmargin=0.5cm,
    numbers=left,
    numbersep=5pt,
    literate={\$}{{\textcolor{blue}{\$}}}1,
    showstringspaces=false,
    upquote=true,
}

\documentclass{article}
\usepackage[margin=0.75in]{geometry}
\usepackage{makecell}
\usepackage{longtable}
\usepackage{amsmath, amssymb}
\usepackage{listings}
\usepackage{xcolor}
\usepackage{tikz}
\usepackage{tcolorbox}

\lstdefinestyle{bashconsole}{
    backgroundcolor=\color{black!5},
    basicstyle=\ttfamily\color{black},
    keywordstyle=\color{blue}\bfseries,
    commentstyle=\color{green!70!black},
    stringstyle=\color{red!70!black},
    numberstyle=\tiny\color{gray},
    breaklines=true,
    frame=single,
    rulecolor=\color{black!20},
    xleftmargin=0.5cm,
    numbers=left,
    numbersep=5pt,
    literate={\$}{{\textcolor{blue}{\$}}}1,
    showstringspaces=false,
    upquote=true,
}

\documentclass{article}
\usepackage[margin=0.75in]{geometry}
\usepackage{makecell}
\usepackage{longtable}
\usepackage{amsmath, amssymb}
\usepackage{listings}
\usepackage{xcolor}
\usepackage{tikz}
\usepackage{tcolorbox}

\lstdefinestyle{bashconsole}{
    backgroundcolor=\color{black!5},
    basicstyle=\ttfamily\color{black},
    keywordstyle=\color{blue}\bfseries,
    commentstyle=\color{green!70!black},
    stringstyle=\color{red!70!black},
    numberstyle=\tiny\color{gray},
    breaklines=true,
    frame=single,
    rulecolor=\color{black!20},
    xleftmargin=0.5cm,
    numbers=left,
    numbersep=5pt,
    literate={\$}{{\textcolor{blue}{\$}}}1,
    showstringspaces=false,
    upquote=true,
}

\documentclass{article}
\usepackage[margin=0.75in]{geometry}
\usepackage{makecell}
\usepackage{longtable}
\usepackage{amsmath, amssymb}
\usepackage{listings}
\usepackage{xcolor}
\usepackage{tikz}
\usepackage{tcolorbox}

\include{styles/bashconsole}
\include{styles/language}

\setlength{\parindent}{0pt}

\title{Language Documentation}
\author{Ruben Saunders}
\date{September 2024}

\begin{document}

\maketitle
\tableofcontents

\newpage

\section{Overview}

The compiler takes one or more source files and produces single, linked assembly file.

\subsection{Command-Line Interface}

The compiler executable is called as follows:

\medskip
\begin{lstlisting}[style=bashconsole]
$ ./compiler <input_file> -o <output_file> [flags]
\end{lstlisting}

The output file is provided after the \texttt{-o} flag.
The following optional flags are available:
\begin{itemize}
    \item \texttt{-d}: enables debug mode.
    In this mode, detailed results from each step are output to the console.
    \item \texttt{-l <file>}: for each file, writes lexed source to \texttt{file} as XML.
    \item \texttt{-p <file>}: for each file, writes parsed contents to \texttt{file} as XML.
\end{itemize}

\subsection{Process Flow}

The process flow bears resemblance to the assembler, and hence will be summarised in less detail.

\begin{enumerate}
    \item For each input file:
    \begin{enumerate}
        \item Open the file.
        \item Pre-process the file.
        \item Parse the file, construct an AST.
        \item Create symbol table.
    \end{enumerate}
    \item Link referenced but undefined symbols using all symbol tables.
    \item Compile components to assembly.
    \item Write all components to assembly output.
\end{enumerate}

\section{General Syntax}

Each file is parsed into tokens, which are then grouped in recognisable sequences.
Tokens are grouped into lines; lines are separated by: a newline, unless the previous token was an operator or the parser is in a bracketed group `\texttt{(...)}';
or a semicolon `\texttt{;}'.

Each file has a global scope, which may only consist of global variables, functions, and data definitions.
Below are outlined key points about syntax; specifics will be documented later.

\begin{itemize}
    \item Single-line commends have the form \texttt{// ...} and last until a newline character is encountered.
    \item Multi-line commands have the form \texttt{/* ... */}.
    \item Identifiers start with a lowercase character which may then be followed by any number of characters, numbers, or underscores.
    That is, \texttt{[a-z][a-zA-Z0-9\_]*}.
    \item Datatype identifiers start with an uppercase character which may then be followed by any number of characters, numbers, or underscores.
    That is, \texttt{[A-Z][a-zA-Z0-9\_]*}.
\end{itemize}

\section{Language Options}

These options are provided to configure the language, enforce syntax, or modify reporting.
They are listed internally, but currently cannot be changed without editing \texttt{src/LanguageOptions.cpp}.

Some options take the form of a reporting level.
The possible values are:
\begin{itemize}
    \item \texttt{-1} -- hidden; disables the reporting.
    \item \texttt{0} -- notice.
    \item \texttt{1} -- warning.
    \item \texttt{2} -- error; compilation is halted.
\end{itemize}

\subsection{\texttt{allow\_alter\_entry}}

\textbf{Default: \texttt{true}}

Enables use of the \texttt{entry} keyword to alter the program's entry point.

\subsection{\texttt{allow\_shadowing}}

\textbf{Default: \texttt{true}}

Enables variable shadowing, wherein existing variables may be re-defined in the same scope.

\begin{lstlisting}[language=CustomLang]
decl a: byte
// ...
decl a: word
\end{lstlisting}

\subsection{\texttt{must\_declare\_functions}}

\textbf{Default: \texttt{false}}

Enforces the declaration of a function signature prior to its definition.
That is, all \texttt{func ...} must be preceded by a matching \texttt{decl func ...}.

\begin{lstlisting}[language=CustomLang]
decl func add(int, int) -> int
// ...
func add(x: int, y: int) -> int { ... }
\end{lstlisting}

\section{Types}

Each variable has a type, indicating tha form, size, and purpose of some data.

\paragraph*{Type Coercion}
This refers to how a value takes on a type given context, and occurs implicitly and only if necessary.
For example, the literal \texttt{42} could take on different types depending on the variable's type.
Another example would be adding two integers of different types; the smaller is coerced into the larger type.

\paragraph*{Type Casting}
This is the explicit conversion of data between two types.
Examples would be: down-sizing an integer value, changing a pointer type.
This is done by preceding a variable or value by a bracketed type.
For example,

\begin{lstlisting}[language=CustomLang]
decl pi: float = 3.14159
decl pi_approx: int = (int) pi
\end{lstlisting}

\subsection{Primitive Types}

These types are built-in to the compiler.

\begin{itemize}
    \item \texttt{byte} -- represents an unsigned 8-bit integer.
    \item \texttt{int} -- represents a signed 32-bit integer.
    \item \texttt{uint} -- represents an unsigned 32-bit integer.
    \item \texttt{word} -- represents a signed 64-bit integer (a processor word).
    \item \texttt{uword} -- represents an unsigned 64-bit integer (a processor word).
    \item \texttt{float} -- represents a 32-bit floating-point number.
    \item \texttt{double} -- represents a 64-bit floating-point number.
\end{itemize}

\paragraph*{Numeric Literals}
Numbers are specified as a sequence of digits.
A different base may be specified by prefixing the literal with \texttt{0}\(x\) where \(x\) is one of

\begin{itemize}
    \item \texttt{b} -- binary, base-2.
    \item \texttt{o} -- octal, base-8.
    \item \texttt{d} -- decimal, base-10 (the default).
    \item \texttt{x} -- hexadecimal, base-16.
\end{itemize}

By default, integer literals are \texttt{int}s, unless the value exceeds the integer capacity, or the literal has a \texttt{w} suffix.

\paragraph*{Decimal Literals}

A numeric literal becomes a float if a decimal point `\texttt{.}' is encountered.

For floating points, the default is \texttt{double} unless \texttt{f} is suffixed.

\subsection{User-Defined Types}

These are types defined using the \texttt{data} keyword, with the syntax

\begin{lstlisting}[language=CustomLang]
decl data Name
data Name {
    field1: type1,
    ...
}
\end{lstlisting}

That is, the datatype \texttt{Name} contains the listed fields, which are the listed types.
Field declarations are separated by newlines, or by commas.

\paragraph*{Member Access}

Members may then be accessed via the dot `\texttt{.}' operator.

\begin{lstlisting}[language=CustomLang]
data Vec { x: int, y: int }
decl v: Vec
v.x // => 0
\end{lstlisting}

\paragraph*{As Parameters}

Variables of a user-defined datatype are passed around as references, meaning that modifications to said parameters modify the original.
For example,

\begin{lstlisting}[language=CustomLang]
func set(v: Vec, n: int) {
    v.x = n
    v.y = n
}

decl v: Vec // v.x = 0
set(v, 5) // v.x = 5
\end{lstlisting}

\paragraph*{Casting}

Values of user-defined datatypes cannot be cast between eachother.
If necessary, cast to a pointer type first.

\begin{lstlisting}[language=CustomLang]
decl data Vec2, Vec3
decl v: Vec2
// ...
decl v: Vec3 = *(*Vec3) @v
\end{lstlisting}

\subsection{Pointers}

Pointers are special types which contain the memory address (location) of a variable of some type.
Pointers are declared with a star, followed by the datatype at the location.
For example,

\begin{lstlisting}[language=CustomLang]
decl n: int = 42
decl p: *int = @n
\end{lstlisting}

If the type is unknown, one uses the special \texttt{*unknown} type.
Note that pointers themselves are nothing but integers under the hood, specifically a \texttt{uword}.

\paragraph*{Creating Pointers}
The memory location of a variable may be retrieved using the address-of `\texttt{@}' unary operator.
Note, such operators are evaluated at compile-time.
As seen above, the variable \(t\) produces pointer \texttt{*\(t\)}.

\paragraph*{Pointer Arithmetic}
Pointers supports both the addition `\texttt{+}' and subtraction `\texttt{-}' operators with integers.
Such operations are considered to intend ``move the pointer left/right by \(n\) units'', the unit size dependent on the pointer type.

\begin{lstlisting}[language=CustomLang]
decl p: *int = 10
p + 2 // p + 2 * sizeof(int) = 18
// vs
decl p: *byte = 10
p + 2 // p + 2 * sizeof(byte) = 12
\end{lstlisting}

\paragraph*{Pointer Dereferencing}
This refers to retrieving a value at a pointer, and is done via the star `\texttt{*}' unary operator.
As expected of the complement of \texttt{@}, this strips a star from the type, and hence is only applied to pointer types.
Following the example, the value of \texttt{n} may be recovered by

\begin{lstlisting}[language=CustomLang]
decl n2: int = *p
\end{lstlisting}

\paragraph*{Casting}
As pointers are but integers, they may be cast as such.
That is, integers may be cast to pointers with any number of stars, and vice versa.

\begin{lstlisting}[language=CustomLang]
decl n: int = 5,
    pint: *int = n,
    pfloat: *float = pint // would also work with `= n'
\end{lstlisting}

\subsection{Arrays}

Arrays are contiguous blocks of memory which may hold a sequence of data of one type.
Essentially, arrays are pointers, except \texttt{sizeof} returns the size in bytes of the array, not of the pointer type.

An array type is declared by suffixing the type with square brackets `\texttt{[]}'.
Note that the pointer specifier `\texttt{*}' is more binding than the array specifier.
That is, \texttt{*int[]} is a pointer to an array of integers, whereas \texttt{(*int)[]} is an array of integer pointers.

The array size is optionally given between the square brackets.

\begin{lstlisting}[language=CustomLang]
decl nums: int[5]
sizeof(nums) // -> 5 * sizeof(int) = 20
\end{lstlisting}

Note that an arrays size must be known at compile time (e.g, a macro or a constant with known value).
If a size is not specified, the declaration \textbf{must} be initialised, from which the size will be deduced.

\begin{lstlisting}[language=CustomLang]
decl nums: int[] = { 1, 2, 3, 4, 5 }
sizeof(nums) // -> 5 * sizeof(int) = 20
\end{lstlisting}

\subsection{Constants}

If a type is preceded by the \texttt{const} keyword, this type is marked as constant and any attempted changes to this type is forbidden.
Additionally, attempting to strip a \texttt{const} type of its constant status is forbidden (but copying to a non-constant is allowed).

\begin{lstlisting}[language=CustomLang]
decl pi: const float = 3.14159
pi = 5 // error! `pi' is marked const
decl pi: float = pi
pi = 5 // permitted, as shadow is not const
\end{lstlisting}

\section{Variables}

Variables are but labels to reserved location in memory.
When defined, variables are assigned a name and a datatype, which dictates the size in bytes of the reserved location.
An example would be

\begin{lstlisting}[language=CustomLang]
decl x: int
\end{lstlisting}

Values may be assigned to variables using the assignment operator `\texttt{=}'.
Note the type coercion/casting behaviours described previously.

\subsection{Multi-Declaration}

Commas may be used to separate declarations and, hence, declare multiple symbols per keyword.
Each declaration may be of a different type.

\begin{lstlisting}[language=CustomLang]
decl a: byte , b: int , c: word
\end{lstlisting}

\subsection{Scope}

``Scope'' refers to where a variable exists.
The global scope is the top-most scope where all top-level functions and variable reside.
Symbols in the global scope may be accessed anywhere in the program.

On the other hand, local scope is not all-encompassing.
A new local scope is introduced in every block.
Variables defined in such a scope are only accessible from within that function; referencing them outside will result in an error.

When a variable is referenced, the scopes will be searched as a stack; that is, local first, global last.

\begin{lstlisting}[language=CustomLang]
decl n: int = 0

func f1 {
    n++ // this will increment the global `n'
}

func f2 {
    decl n: int = 1
    n++ // this will increment the local `n' declared above
}
\end{lstlisting}

To see an example of creating a local scope that is not a function definition:

\begin{lstlisting}[language=CustomLang]
decl n: int = 0 // n = 0

{
    decl n: int = 2 // n = 2
    n++ // n = 3
}

// n = 0
\end{lstlisting}

\section{Functions}

Functions are name-associated sections of code which may be called, possible with arguments, and may return a value.
They are defined using the \texttt{func} keyword.
For use before definitions, signatures may be declared using the compound \texttt{decl func} keyword.

\begin{lstlisting}[language=CustomLang]
decl func add(int, int) -> int
func add(a: int, b: int) {
    return a + b
}
\end{lstlisting}

\begin{itemize}
    \item In declarations, parameter names are not required.
    \item If no parameters are required, it is possible to omit the brackets entirely.
    \item If no return type is required, omit the arrow `\texttt{->}' and the type.
    \item If declared, the definition does not require a return type as this can be inferred from its declared signature.
\end{itemize}

\subsection{Overloading}

Function overloading is supported, meaning that a function name may be re-used with a different signature.
For example,

\begin{lstlisting}[language=CustomLang]
decl func add(int, int) -> int
decl func add(float, float) -> float
\end{lstlisting}

\subsection{Entry Point}

All programs have an entry point.
By default, it is \texttt{main}, taking zero or more integers, and optionally returning an integer.

A new entry point may be defined using the \texttt{entry} keyword, by following the keyword by the entry point's name and type.
Note, this is a function signature.

\begin{lstlisting}[language=CustomLang]
entry start(int) -> int
\end{lstlisting}

Only one entry point per program is permitted.
Once encountered, future encounters of \texttt{entry} will result in an error.

\subsection{Compile-Time Functions}

These ``functions'' are resolved in the compilation stage.

\paragraph*{\texttt{sizeof(\(t\))}}
This returns the size, in bytes, of the argument \(t\).
\(t\) may be a type name, or a variable, in which case the size of the variable's type will be calculated.

\begin{lstlisting}[language=CustomLang]
sizeof(int) // -> 4
\end{lstlisting}

\paragraph*{\texttt{register(\(r\))}}
This may be used in expressions, and returns the contents of register \(r\) as a word.
\(r\) is the name of a register, same as in the assembly code but without the dollar `\texttt{\$}'.

\begin{lstlisting}[language=CustomLang]
register(sp) // reads $sp
\end{lstlisting}

\section{Operators}

\textit{TODO}

\end{document}


\setlength{\parindent}{0pt}

\title{Language Documentation}
\author{Ruben Saunders}
\date{September 2024}

\begin{document}

\maketitle
\tableofcontents

\newpage

\section{Overview}

The compiler takes one or more source files and produces single, linked assembly file.

\subsection{Command-Line Interface}

The compiler executable is called as follows:

\medskip
\begin{lstlisting}[style=bashconsole]
$ ./compiler <input_file> -o <output_file> [flags]
\end{lstlisting}

The output file is provided after the \texttt{-o} flag.
The following optional flags are available:
\begin{itemize}
    \item \texttt{-d}: enables debug mode.
    In this mode, detailed results from each step are output to the console.
    \item \texttt{-l <file>}: for each file, writes lexed source to \texttt{file} as XML.
    \item \texttt{-p <file>}: for each file, writes parsed contents to \texttt{file} as XML.
\end{itemize}

\subsection{Process Flow}

The process flow bears resemblance to the assembler, and hence will be summarised in less detail.

\begin{enumerate}
    \item For each input file:
    \begin{enumerate}
        \item Open the file.
        \item Pre-process the file.
        \item Parse the file, construct an AST.
        \item Create symbol table.
    \end{enumerate}
    \item Link referenced but undefined symbols using all symbol tables.
    \item Compile components to assembly.
    \item Write all components to assembly output.
\end{enumerate}

\section{General Syntax}

Each file is parsed into tokens, which are then grouped in recognisable sequences.
Tokens are grouped into lines; lines are separated by: a newline, unless the previous token was an operator or the parser is in a bracketed group `\texttt{(...)}';
or a semicolon `\texttt{;}'.

Each file has a global scope, which may only consist of global variables, functions, and data definitions.
Below are outlined key points about syntax; specifics will be documented later.

\begin{itemize}
    \item Single-line commends have the form \texttt{// ...} and last until a newline character is encountered.
    \item Multi-line commands have the form \texttt{/* ... */}.
    \item Identifiers start with a lowercase character which may then be followed by any number of characters, numbers, or underscores.
    That is, \texttt{[a-z][a-zA-Z0-9\_]*}.
    \item Datatype identifiers start with an uppercase character which may then be followed by any number of characters, numbers, or underscores.
    That is, \texttt{[A-Z][a-zA-Z0-9\_]*}.
\end{itemize}

\section{Language Options}

These options are provided to configure the language, enforce syntax, or modify reporting.
They are listed internally, but currently cannot be changed without editing \texttt{src/LanguageOptions.cpp}.

Some options take the form of a reporting level.
The possible values are:
\begin{itemize}
    \item \texttt{-1} -- hidden; disables the reporting.
    \item \texttt{0} -- notice.
    \item \texttt{1} -- warning.
    \item \texttt{2} -- error; compilation is halted.
\end{itemize}

\subsection{\texttt{allow\_alter\_entry}}

\textbf{Default: \texttt{true}}

Enables use of the \texttt{entry} keyword to alter the program's entry point.

\subsection{\texttt{allow\_shadowing}}

\textbf{Default: \texttt{true}}

Enables variable shadowing, wherein existing variables may be re-defined in the same scope.

\begin{lstlisting}[language=CustomLang]
decl a: byte
// ...
decl a: word
\end{lstlisting}

\subsection{\texttt{must\_declare\_functions}}

\textbf{Default: \texttt{false}}

Enforces the declaration of a function signature prior to its definition.
That is, all \texttt{func ...} must be preceded by a matching \texttt{decl func ...}.

\begin{lstlisting}[language=CustomLang]
decl func add(int, int) -> int
// ...
func add(x: int, y: int) -> int { ... }
\end{lstlisting}

\section{Types}

Each variable has a type, indicating tha form, size, and purpose of some data.

\paragraph*{Type Coercion}
This refers to how a value takes on a type given context, and occurs implicitly and only if necessary.
For example, the literal \texttt{42} could take on different types depending on the variable's type.
Another example would be adding two integers of different types; the smaller is coerced into the larger type.

\paragraph*{Type Casting}
This is the explicit conversion of data between two types.
Examples would be: down-sizing an integer value, changing a pointer type.
This is done by preceding a variable or value by a bracketed type.
For example,

\begin{lstlisting}[language=CustomLang]
decl pi: float = 3.14159
decl pi_approx: int = (int) pi
\end{lstlisting}

\subsection{Primitive Types}

These types are built-in to the compiler.

\begin{itemize}
    \item \texttt{byte} -- represents an unsigned 8-bit integer.
    \item \texttt{int} -- represents a signed 32-bit integer.
    \item \texttt{uint} -- represents an unsigned 32-bit integer.
    \item \texttt{word} -- represents a signed 64-bit integer (a processor word).
    \item \texttt{uword} -- represents an unsigned 64-bit integer (a processor word).
    \item \texttt{float} -- represents a 32-bit floating-point number.
    \item \texttt{double} -- represents a 64-bit floating-point number.
\end{itemize}

\paragraph*{Numeric Literals}
Numbers are specified as a sequence of digits.
A different base may be specified by prefixing the literal with \texttt{0}\(x\) where \(x\) is one of

\begin{itemize}
    \item \texttt{b} -- binary, base-2.
    \item \texttt{o} -- octal, base-8.
    \item \texttt{d} -- decimal, base-10 (the default).
    \item \texttt{x} -- hexadecimal, base-16.
\end{itemize}

By default, integer literals are \texttt{int}s, unless the value exceeds the integer capacity, or the literal has a \texttt{w} suffix.

\paragraph*{Decimal Literals}

A numeric literal becomes a float if a decimal point `\texttt{.}' is encountered.

For floating points, the default is \texttt{double} unless \texttt{f} is suffixed.

\subsection{User-Defined Types}

These are types defined using the \texttt{data} keyword, with the syntax

\begin{lstlisting}[language=CustomLang]
decl data Name
data Name {
    field1: type1,
    ...
}
\end{lstlisting}

That is, the datatype \texttt{Name} contains the listed fields, which are the listed types.
Field declarations are separated by newlines, or by commas.

\paragraph*{Member Access}

Members may then be accessed via the dot `\texttt{.}' operator.

\begin{lstlisting}[language=CustomLang]
data Vec { x: int, y: int }
decl v: Vec
v.x // => 0
\end{lstlisting}

\paragraph*{As Parameters}

Variables of a user-defined datatype are passed around as references, meaning that modifications to said parameters modify the original.
For example,

\begin{lstlisting}[language=CustomLang]
func set(v: Vec, n: int) {
    v.x = n
    v.y = n
}

decl v: Vec // v.x = 0
set(v, 5) // v.x = 5
\end{lstlisting}

\paragraph*{Casting}

Values of user-defined datatypes cannot be cast between eachother.
If necessary, cast to a pointer type first.

\begin{lstlisting}[language=CustomLang]
decl data Vec2, Vec3
decl v: Vec2
// ...
decl v: Vec3 = *(*Vec3) @v
\end{lstlisting}

\subsection{Pointers}

Pointers are special types which contain the memory address (location) of a variable of some type.
Pointers are declared with a star, followed by the datatype at the location.
For example,

\begin{lstlisting}[language=CustomLang]
decl n: int = 42
decl p: *int = @n
\end{lstlisting}

If the type is unknown, one uses the special \texttt{*unknown} type.
Note that pointers themselves are nothing but integers under the hood, specifically a \texttt{uword}.

\paragraph*{Creating Pointers}
The memory location of a variable may be retrieved using the address-of `\texttt{@}' unary operator.
Note, such operators are evaluated at compile-time.
As seen above, the variable \(t\) produces pointer \texttt{*\(t\)}.

\paragraph*{Pointer Arithmetic}
Pointers supports both the addition `\texttt{+}' and subtraction `\texttt{-}' operators with integers.
Such operations are considered to intend ``move the pointer left/right by \(n\) units'', the unit size dependent on the pointer type.

\begin{lstlisting}[language=CustomLang]
decl p: *int = 10
p + 2 // p + 2 * sizeof(int) = 18
// vs
decl p: *byte = 10
p + 2 // p + 2 * sizeof(byte) = 12
\end{lstlisting}

\paragraph*{Pointer Dereferencing}
This refers to retrieving a value at a pointer, and is done via the star `\texttt{*}' unary operator.
As expected of the complement of \texttt{@}, this strips a star from the type, and hence is only applied to pointer types.
Following the example, the value of \texttt{n} may be recovered by

\begin{lstlisting}[language=CustomLang]
decl n2: int = *p
\end{lstlisting}

\paragraph*{Casting}
As pointers are but integers, they may be cast as such.
That is, integers may be cast to pointers with any number of stars, and vice versa.

\begin{lstlisting}[language=CustomLang]
decl n: int = 5,
    pint: *int = n,
    pfloat: *float = pint // would also work with `= n'
\end{lstlisting}

\subsection{Arrays}

Arrays are contiguous blocks of memory which may hold a sequence of data of one type.
Essentially, arrays are pointers, except \texttt{sizeof} returns the size in bytes of the array, not of the pointer type.

An array type is declared by suffixing the type with square brackets `\texttt{[]}'.
Note that the pointer specifier `\texttt{*}' is more binding than the array specifier.
That is, \texttt{*int[]} is a pointer to an array of integers, whereas \texttt{(*int)[]} is an array of integer pointers.

The array size is optionally given between the square brackets.

\begin{lstlisting}[language=CustomLang]
decl nums: int[5]
sizeof(nums) // -> 5 * sizeof(int) = 20
\end{lstlisting}

Note that an arrays size must be known at compile time (e.g, a macro or a constant with known value).
If a size is not specified, the declaration \textbf{must} be initialised, from which the size will be deduced.

\begin{lstlisting}[language=CustomLang]
decl nums: int[] = { 1, 2, 3, 4, 5 }
sizeof(nums) // -> 5 * sizeof(int) = 20
\end{lstlisting}

\subsection{Constants}

If a type is preceded by the \texttt{const} keyword, this type is marked as constant and any attempted changes to this type is forbidden.
Additionally, attempting to strip a \texttt{const} type of its constant status is forbidden (but copying to a non-constant is allowed).

\begin{lstlisting}[language=CustomLang]
decl pi: const float = 3.14159
pi = 5 // error! `pi' is marked const
decl pi: float = pi
pi = 5 // permitted, as shadow is not const
\end{lstlisting}

\section{Variables}

Variables are but labels to reserved location in memory.
When defined, variables are assigned a name and a datatype, which dictates the size in bytes of the reserved location.
An example would be

\begin{lstlisting}[language=CustomLang]
decl x: int
\end{lstlisting}

Values may be assigned to variables using the assignment operator `\texttt{=}'.
Note the type coercion/casting behaviours described previously.

\subsection{Multi-Declaration}

Commas may be used to separate declarations and, hence, declare multiple symbols per keyword.
Each declaration may be of a different type.

\begin{lstlisting}[language=CustomLang]
decl a: byte , b: int , c: word
\end{lstlisting}

\subsection{Scope}

``Scope'' refers to where a variable exists.
The global scope is the top-most scope where all top-level functions and variable reside.
Symbols in the global scope may be accessed anywhere in the program.

On the other hand, local scope is not all-encompassing.
A new local scope is introduced in every block.
Variables defined in such a scope are only accessible from within that function; referencing them outside will result in an error.

When a variable is referenced, the scopes will be searched as a stack; that is, local first, global last.

\begin{lstlisting}[language=CustomLang]
decl n: int = 0

func f1 {
    n++ // this will increment the global `n'
}

func f2 {
    decl n: int = 1
    n++ // this will increment the local `n' declared above
}
\end{lstlisting}

To see an example of creating a local scope that is not a function definition:

\begin{lstlisting}[language=CustomLang]
decl n: int = 0 // n = 0

{
    decl n: int = 2 // n = 2
    n++ // n = 3
}

// n = 0
\end{lstlisting}

\section{Functions}

Functions are name-associated sections of code which may be called, possible with arguments, and may return a value.
They are defined using the \texttt{func} keyword.
For use before definitions, signatures may be declared using the compound \texttt{decl func} keyword.

\begin{lstlisting}[language=CustomLang]
decl func add(int, int) -> int
func add(a: int, b: int) {
    return a + b
}
\end{lstlisting}

\begin{itemize}
    \item In declarations, parameter names are not required.
    \item If no parameters are required, it is possible to omit the brackets entirely.
    \item If no return type is required, omit the arrow `\texttt{->}' and the type.
    \item If declared, the definition does not require a return type as this can be inferred from its declared signature.
\end{itemize}

\subsection{Overloading}

Function overloading is supported, meaning that a function name may be re-used with a different signature.
For example,

\begin{lstlisting}[language=CustomLang]
decl func add(int, int) -> int
decl func add(float, float) -> float
\end{lstlisting}

\subsection{Entry Point}

All programs have an entry point.
By default, it is \texttt{main}, taking zero or more integers, and optionally returning an integer.

A new entry point may be defined using the \texttt{entry} keyword, by following the keyword by the entry point's name and type.
Note, this is a function signature.

\begin{lstlisting}[language=CustomLang]
entry start(int) -> int
\end{lstlisting}

Only one entry point per program is permitted.
Once encountered, future encounters of \texttt{entry} will result in an error.

\subsection{Compile-Time Functions}

These ``functions'' are resolved in the compilation stage.

\paragraph*{\texttt{sizeof(\(t\))}}
This returns the size, in bytes, of the argument \(t\).
\(t\) may be a type name, or a variable, in which case the size of the variable's type will be calculated.

\begin{lstlisting}[language=CustomLang]
sizeof(int) // -> 4
\end{lstlisting}

\paragraph*{\texttt{register(\(r\))}}
This may be used in expressions, and returns the contents of register \(r\) as a word.
\(r\) is the name of a register, same as in the assembly code but without the dollar `\texttt{\$}'.

\begin{lstlisting}[language=CustomLang]
register(sp) // reads $sp
\end{lstlisting}

\section{Operators}

\textit{TODO}

\end{document}


\setlength{\parindent}{0pt}

\title{Language Documentation}
\author{Ruben Saunders}
\date{September 2024}

\begin{document}

\maketitle
\tableofcontents

\newpage

\section{Overview}

The compiler takes one or more source files and produces single, linked assembly file.

\subsection{Command-Line Interface}

The compiler executable is called as follows:

\medskip
\begin{lstlisting}[style=bashconsole]
$ ./compiler <input_file> -o <output_file> [flags]
\end{lstlisting}

The output file is provided after the \texttt{-o} flag.
The following optional flags are available:
\begin{itemize}
    \item \texttt{-d}: enables debug mode.
    In this mode, detailed results from each step are output to the console.
    \item \texttt{-l <file>}: for each file, writes lexed source to \texttt{file} as XML.
    \item \texttt{-p <file>}: for each file, writes parsed contents to \texttt{file} as XML.
\end{itemize}

\subsection{Process Flow}

The process flow bears resemblance to the assembler, and hence will be summarised in less detail.

\begin{enumerate}
    \item For each input file:
    \begin{enumerate}
        \item Open the file.
        \item Pre-process the file.
        \item Parse the file, construct an AST.
        \item Create symbol table.
    \end{enumerate}
    \item Link referenced but undefined symbols using all symbol tables.
    \item Compile components to assembly.
    \item Write all components to assembly output.
\end{enumerate}

\section{General Syntax}

Each file is parsed into tokens, which are then grouped in recognisable sequences.
Tokens are grouped into lines; lines are separated by: a newline, unless the previous token was an operator or the parser is in a bracketed group `\texttt{(...)}';
or a semicolon `\texttt{;}'.

Each file has a global scope, which may only consist of global variables, functions, and data definitions.
Below are outlined key points about syntax; specifics will be documented later.

\begin{itemize}
    \item Single-line commends have the form \texttt{// ...} and last until a newline character is encountered.
    \item Multi-line commands have the form \texttt{/* ... */}.
    \item Identifiers start with a lowercase character which may then be followed by any number of characters, numbers, or underscores.
    That is, \texttt{[a-z][a-zA-Z0-9\_]*}.
    \item Datatype identifiers start with an uppercase character which may then be followed by any number of characters, numbers, or underscores.
    That is, \texttt{[A-Z][a-zA-Z0-9\_]*}.
\end{itemize}

\section{Language Options}

These options are provided to configure the language, enforce syntax, or modify reporting.
They are listed internally, but currently cannot be changed without editing \texttt{src/LanguageOptions.cpp}.

Some options take the form of a reporting level.
The possible values are:
\begin{itemize}
    \item \texttt{-1} -- hidden; disables the reporting.
    \item \texttt{0} -- notice.
    \item \texttt{1} -- warning.
    \item \texttt{2} -- error; compilation is halted.
\end{itemize}

\subsection{\texttt{allow\_alter\_entry}}

\textbf{Default: \texttt{true}}

Enables use of the \texttt{entry} keyword to alter the program's entry point.

\subsection{\texttt{allow\_shadowing}}

\textbf{Default: \texttt{true}}

Enables variable shadowing, wherein existing variables may be re-defined in the same scope.

\begin{lstlisting}[language=CustomLang]
decl a: byte
// ...
decl a: word
\end{lstlisting}

\subsection{\texttt{must\_declare\_functions}}

\textbf{Default: \texttt{false}}

Enforces the declaration of a function signature prior to its definition.
That is, all \texttt{func ...} must be preceded by a matching \texttt{decl func ...}.

\begin{lstlisting}[language=CustomLang]
decl func add(int, int) -> int
// ...
func add(x: int, y: int) -> int { ... }
\end{lstlisting}

\section{Types}

Each variable has a type, indicating tha form, size, and purpose of some data.

\paragraph*{Type Coercion}
This refers to how a value takes on a type given context, and occurs implicitly and only if necessary.
For example, the literal \texttt{42} could take on different types depending on the variable's type.
Another example would be adding two integers of different types; the smaller is coerced into the larger type.

\paragraph*{Type Casting}
This is the explicit conversion of data between two types.
Examples would be: down-sizing an integer value, changing a pointer type.
This is done by preceding a variable or value by a bracketed type.
For example,

\begin{lstlisting}[language=CustomLang]
decl pi: float = 3.14159
decl pi_approx: int = (int) pi
\end{lstlisting}

\subsection{Primitive Types}

These types are built-in to the compiler.

\begin{itemize}
    \item \texttt{byte} -- represents an unsigned 8-bit integer.
    \item \texttt{int} -- represents a signed 32-bit integer.
    \item \texttt{uint} -- represents an unsigned 32-bit integer.
    \item \texttt{word} -- represents a signed 64-bit integer (a processor word).
    \item \texttt{uword} -- represents an unsigned 64-bit integer (a processor word).
    \item \texttt{float} -- represents a 32-bit floating-point number.
    \item \texttt{double} -- represents a 64-bit floating-point number.
\end{itemize}

\paragraph*{Numeric Literals}
Numbers are specified as a sequence of digits.
A different base may be specified by prefixing the literal with \texttt{0}\(x\) where \(x\) is one of

\begin{itemize}
    \item \texttt{b} -- binary, base-2.
    \item \texttt{o} -- octal, base-8.
    \item \texttt{d} -- decimal, base-10 (the default).
    \item \texttt{x} -- hexadecimal, base-16.
\end{itemize}

By default, integer literals are \texttt{int}s, unless the value exceeds the integer capacity, or the literal has a \texttt{w} suffix.

\paragraph*{Decimal Literals}

A numeric literal becomes a float if a decimal point `\texttt{.}' is encountered.

For floating points, the default is \texttt{double} unless \texttt{f} is suffixed.

\subsection{User-Defined Types}

These are types defined using the \texttt{data} keyword, with the syntax

\begin{lstlisting}[language=CustomLang]
decl data Name
data Name {
    field1: type1,
    ...
}
\end{lstlisting}

That is, the datatype \texttt{Name} contains the listed fields, which are the listed types.
Field declarations are separated by newlines, or by commas.

\paragraph*{Member Access}

Members may then be accessed via the dot `\texttt{.}' operator.

\begin{lstlisting}[language=CustomLang]
data Vec { x: int, y: int }
decl v: Vec
v.x // => 0
\end{lstlisting}

\paragraph*{As Parameters}

Variables of a user-defined datatype are passed around as references, meaning that modifications to said parameters modify the original.
For example,

\begin{lstlisting}[language=CustomLang]
func set(v: Vec, n: int) {
    v.x = n
    v.y = n
}

decl v: Vec // v.x = 0
set(v, 5) // v.x = 5
\end{lstlisting}

\paragraph*{Casting}

Values of user-defined datatypes cannot be cast between eachother.
If necessary, cast to a pointer type first.

\begin{lstlisting}[language=CustomLang]
decl data Vec2, Vec3
decl v: Vec2
// ...
decl v: Vec3 = *(*Vec3) @v
\end{lstlisting}

\subsection{Pointers}

Pointers are special types which contain the memory address (location) of a variable of some type.
Pointers are declared with a star, followed by the datatype at the location.
For example,

\begin{lstlisting}[language=CustomLang]
decl n: int = 42
decl p: *int = @n
\end{lstlisting}

If the type is unknown, one uses the special \texttt{*unknown} type.
Note that pointers themselves are nothing but integers under the hood, specifically a \texttt{uword}.

\paragraph*{Creating Pointers}
The memory location of a variable may be retrieved using the address-of `\texttt{@}' unary operator.
Note, such operators are evaluated at compile-time.
As seen above, the variable \(t\) produces pointer \texttt{*\(t\)}.

\paragraph*{Pointer Arithmetic}
Pointers supports both the addition `\texttt{+}' and subtraction `\texttt{-}' operators with integers.
Such operations are considered to intend ``move the pointer left/right by \(n\) units'', the unit size dependent on the pointer type.

\begin{lstlisting}[language=CustomLang]
decl p: *int = 10
p + 2 // p + 2 * sizeof(int) = 18
// vs
decl p: *byte = 10
p + 2 // p + 2 * sizeof(byte) = 12
\end{lstlisting}

\paragraph*{Pointer Dereferencing}
This refers to retrieving a value at a pointer, and is done via the star `\texttt{*}' unary operator.
As expected of the complement of \texttt{@}, this strips a star from the type, and hence is only applied to pointer types.
Following the example, the value of \texttt{n} may be recovered by

\begin{lstlisting}[language=CustomLang]
decl n2: int = *p
\end{lstlisting}

\paragraph*{Casting}
As pointers are but integers, they may be cast as such.
That is, integers may be cast to pointers with any number of stars, and vice versa.

\begin{lstlisting}[language=CustomLang]
decl n: int = 5,
    pint: *int = n,
    pfloat: *float = pint // would also work with `= n'
\end{lstlisting}

\subsection{Arrays}

Arrays are contiguous blocks of memory which may hold a sequence of data of one type.
Essentially, arrays are pointers, except \texttt{sizeof} returns the size in bytes of the array, not of the pointer type.

An array type is declared by suffixing the type with square brackets `\texttt{[]}'.
Note that the pointer specifier `\texttt{*}' is more binding than the array specifier.
That is, \texttt{*int[]} is a pointer to an array of integers, whereas \texttt{(*int)[]} is an array of integer pointers.

The array size is optionally given between the square brackets.

\begin{lstlisting}[language=CustomLang]
decl nums: int[5]
sizeof(nums) // -> 5 * sizeof(int) = 20
\end{lstlisting}

Note that an arrays size must be known at compile time (e.g, a macro or a constant with known value).
If a size is not specified, the declaration \textbf{must} be initialised, from which the size will be deduced.

\begin{lstlisting}[language=CustomLang]
decl nums: int[] = { 1, 2, 3, 4, 5 }
sizeof(nums) // -> 5 * sizeof(int) = 20
\end{lstlisting}

\subsection{Constants}

If a type is preceded by the \texttt{const} keyword, this type is marked as constant and any attempted changes to this type is forbidden.
Additionally, attempting to strip a \texttt{const} type of its constant status is forbidden (but copying to a non-constant is allowed).

\begin{lstlisting}[language=CustomLang]
decl pi: const float = 3.14159
pi = 5 // error! `pi' is marked const
decl pi: float = pi
pi = 5 // permitted, as shadow is not const
\end{lstlisting}

\section{Variables}

Variables are but labels to reserved location in memory.
When defined, variables are assigned a name and a datatype, which dictates the size in bytes of the reserved location.
An example would be

\begin{lstlisting}[language=CustomLang]
decl x: int
\end{lstlisting}

Values may be assigned to variables using the assignment operator `\texttt{=}'.
Note the type coercion/casting behaviours described previously.

\subsection{Multi-Declaration}

Commas may be used to separate declarations and, hence, declare multiple symbols per keyword.
Each declaration may be of a different type.

\begin{lstlisting}[language=CustomLang]
decl a: byte , b: int , c: word
\end{lstlisting}

\subsection{Scope}

``Scope'' refers to where a variable exists.
The global scope is the top-most scope where all top-level functions and variable reside.
Symbols in the global scope may be accessed anywhere in the program.

On the other hand, local scope is not all-encompassing.
A new local scope is introduced in every block.
Variables defined in such a scope are only accessible from within that function; referencing them outside will result in an error.

When a variable is referenced, the scopes will be searched as a stack; that is, local first, global last.

\begin{lstlisting}[language=CustomLang]
decl n: int = 0

func f1 {
    n++ // this will increment the global `n'
}

func f2 {
    decl n: int = 1
    n++ // this will increment the local `n' declared above
}
\end{lstlisting}

To see an example of creating a local scope that is not a function definition:

\begin{lstlisting}[language=CustomLang]
decl n: int = 0 // n = 0

{
    decl n: int = 2 // n = 2
    n++ // n = 3
}

// n = 0
\end{lstlisting}

\section{Functions}

Functions are name-associated sections of code which may be called, possible with arguments, and may return a value.
They are defined using the \texttt{func} keyword.
For use before definitions, signatures may be declared using the compound \texttt{decl func} keyword.

\begin{lstlisting}[language=CustomLang]
decl func add(int, int) -> int
func add(a: int, b: int) {
    return a + b
}
\end{lstlisting}

\begin{itemize}
    \item In declarations, parameter names are not required.
    \item If no parameters are required, it is possible to omit the brackets entirely.
    \item If no return type is required, omit the arrow `\texttt{->}' and the type.
    \item If declared, the definition does not require a return type as this can be inferred from its declared signature.
\end{itemize}

\subsection{Overloading}

Function overloading is supported, meaning that a function name may be re-used with a different signature.
For example,

\begin{lstlisting}[language=CustomLang]
decl func add(int, int) -> int
decl func add(float, float) -> float
\end{lstlisting}

\subsection{Entry Point}

All programs have an entry point.
By default, it is \texttt{main}, taking zero or more integers, and optionally returning an integer.

A new entry point may be defined using the \texttt{entry} keyword, by following the keyword by the entry point's name and type.
Note, this is a function signature.

\begin{lstlisting}[language=CustomLang]
entry start(int) -> int
\end{lstlisting}

Only one entry point per program is permitted.
Once encountered, future encounters of \texttt{entry} will result in an error.

\subsection{Compile-Time Functions}

These ``functions'' are resolved in the compilation stage.

\paragraph*{\texttt{sizeof(\(t\))}}
This returns the size, in bytes, of the argument \(t\).
\(t\) may be a type name, or a variable, in which case the size of the variable's type will be calculated.

\begin{lstlisting}[language=CustomLang]
sizeof(int) // -> 4
\end{lstlisting}

\paragraph*{\texttt{register(\(r\))}}
This may be used in expressions, and returns the contents of register \(r\) as a word.
\(r\) is the name of a register, same as in the assembly code but without the dollar `\texttt{\$}'.

\begin{lstlisting}[language=CustomLang]
register(sp) // reads $sp
\end{lstlisting}

\section{Operators}

\textit{TODO}

\end{document}


\setlength{\parindent}{0pt}

\title{Language Documentation}
\author{Ruben Saunders}
\date{January 2025}

\begin{document}
    \maketitle
    \tableofcontents

    \newpage

    \section{Introduction}\label{sec:introduction}

    This document will contain information about the Edel language.
    As an educational language, it is designed to be simple and easy to learn and operate, with the goals of being clear a readable.
    Each section is written from the perspective of little programming knowledge.

    \subsection{Overview}\label{subsec:overview}

    As the most wide-spread paradigm, this language is imperative by nature.
    Its syntax is inspired by C, but borrows features from C++ such as namespaces and operator overloading.
    However, unlike these two languages, names (except variables) are \textit{forward declared}, meaning they can be used prior to their definition.
    This foregoes the need for header files of explicit forward declarations as in C and C++, and allows a much freer programming experience similar to modern languages.

    % TODO talk about imports

    \subsection{The Compiler}\label{subsec:the-compiler}

    A program consists of one file (the entry file), which is compiled by the compiler to produce an assembly output.

    \medskip
    \begin{lstlisting}[style=bashconsole]
$ ./compiler <input_file> [-o <output_file>] [flags]
    \end{lstlisting}
    \medskip

    Where \texttt{[flags]} are as follows:
    \begin{itemize}
        \item \texttt{--ast} - prints the parsed program's AST.
        \item \texttt{--[no-]function-placeholder} - changes behaviour of a function declaration; see subsection ~\ref{subsec:function-declaration}.
        \item \texttt{--[no-]indentation} - adds/removes indentation from generated assembly.
    \end{itemize}
    Note that the output file is optional.

    \subsection{``Hello, World''}\label{subsec:hello-world}

    As is customary, a quick ``hello world'' program written in Edel.

    \begin{lstlisting}[language=CustomLang]
func main() {
    print("Hello, world\n");
}
    \end{lstlisting}

    \section{Scope}

    A \textit{name} in Edel refers to an identifier, which includes functions, variables, and namespaces.
    All names have a \textit{scope}, which is the area in which that name exists/may be referenced.
    That is, outside its scope, a symbol does not exist.

    The outermost scope is known as the \textit{global} scope.
    Names defined here will be available anywhere in the current file.

    The counterpart is the \textit{local} scope, which refers to the latest scope.
    New scopes are created by using code blocks, for example, in while loops and functions.
    Subsection ~\ref{subsec:variables} covers variable definitions and scope in more detail.



    \section{Basic Syntax}\label{sec:basic-syntax}

    Inspires by C, Edel is not whitespace sensitive, with scopes marked by braces \texttt{\{\}}.
    Semicolons are not requires after every line, namely closing braces, but are required after expressions.

    Comments come in two forms: single-- and multi-line.
    Single-line comments \texttt{// ...} comment out the remainder of the current line;
    multi-line comments \texttt{/* ... */} comment out anything they enclose.

    \subsection{Variables}\label{subsec:variables}

    Using the common analogy, a variable is like a labeled box which stored data.
    Each box has a \textit{type} which tells us what it stores.
    For example, ``int 3''.
    More technically, a variable is a location in memory where data can be stored, with its type determining the size of the variable and how its value can be used.

    Variables are defined using the \texttt{let} keyword, with the variables type following a colon.
    For example, to define our ``int 3'' from earlier,
    \begin{lstlisting}[language=CustomLang]
let foo: int = 3;
    \end{lstlisting}

    If required, multiple variables can be defined in the same \texttt{let} statement.
    \begin{lstlisting}[language=CustomLang]
let foo: int = 3, bar: float = 3.14;
    \end{lstlisting}

    If the type is omitted, it is deduced from the assigned value.
    Hence, this can only be done if the variable is assigned.
    \begin{lstlisting}[language=CustomLang]
let foo = 3, bar = 3.14;
    \end{lstlisting}
    (In this case, both variables adopt the types as previously written.)

    \subsubsection{Shadowing}

    Edel allows names to be \textit{shadowed}, meaning a variable may re-use a variable's name.

    In non-local shadowing, the previous definition becomes visible again after the current scope is exited.
    \begin{lstlisting}[language=CustomLang]
let foo = 1;
{
    let foo = 2;
    // foo = 2
}
// foo = 1
    \end{lstlisting}

    Edel also allows local shadowing.
    In this case, once shadowed, the previous definition is essentially lost and cannot be referenced.
    \begin{lstlisting}[language=CustomLang]
let foo = 1;
// foo = 1
let foo = 2;
// foo = 2
    \end{lstlisting}

    \subsubsection{Constants}

    Constants are values that, once defined, cannot be changed.
    Constants may be defined using the \texttt{const} keyword in place of \texttt{let}.

    \begin{lstlisting}[language=CustomLang]
const pi = 3.14159;
    \end{lstlisting}

    In the case of multiple definitions, \textit{all} symbols will be defined as const.

    \subsection{Expressions \& Statements}\label{subsec:expressions-&-statements}

    Generally, a statement is some code which does not produce a value, whereas an expression does.
    In Edel, \textit{everything} is an expression, with statements, by default, returning the unit type `\texttt{()}'.

    Expressions are expected to be terminated by a semicolon.
    The exception to this is the final expression in a block; if omitted, the block `returns' the result of this expression.

    \begin{lstlisting}[language=CustomLang]
let a = { 1 + 2; }; // a = ()
let b = { 1 + 2 }; // b = 3
let c = { let z = 2 * b; z } + 2; // c = 8
    \end{lstlisting}

    Note that only the last expression in a block may have their semicolon emitted.
    By always terminating expressions, the block will act like a statement.

    \subsubsection{Value Categories}

    Each expression, be it an operator with operands, literals, or variable names etc., is characterised by a \textit{type} and a \textit{category}.
    Types, covered in section ~\ref{sec:types}, give meaning to data and what it means in context, whereas the category tells dictates \textit{how} a value can be used.

    Edel has two categories, be them l-- and r-values.
    The former is a value which refers to a memory location, such as variables, while the latter has no identifiable location in memory.
    The \textit{l} and \textit{r} stand for \textit{left} and \textit{right} because the respective values may appear on the left-- or right-hand side of an assignment operator.
    That is,

    \begin{lstlisting}[language=CustomLang]
a = 1; // a (lvalue) = 1 (rvalue)
    \end{lstlisting}

    As an rvalue has no identifiable location, it cannot be assigned to, nor can it be access via the dot `\texttt{.}' operator.
    Instances of misuse of an l-- or rvalue results in an error ``\texttt{expected l/rvalue, got type}''.

    \begin{lstlisting}[language=CustomLang]
1 = 2; // error: expected lvalue, got i32
true.a; // error: expected lvalue, got bool

namespace maths {}
maths + 1; // error: expected rvalue, got namespace
    \end{lstlisting}

    \subsection{Code Blocks}\label{subsec:code-blocks}

    As a whitespace-ignorant language, Edel uses blocks to denote structure and scope.
    A block is introduced by braces \texttt{\{\}} and contains a section of code.
    As previously mentioned, a block may be an expression or a statement, hence is permitted anywhere they would be expected.

    Blocks in Edel introduce a new lexical scope (see section ~\ref{sec:scope} for scope resolution).

    \section{Types}\label{sec:types}

    Continuing the analogy from variables, a type is a label that tells the program what kind of data a variable holds.
    Every variable and value has a type; a value without a type is meaningless, as a type gives data a meaning.

    \subsection{Primitive Types}

    Primitive types are a subset of those provided by the compiler representing atomic values, such as numbers.

    \medskip
    \begin{table}[H]
        \centering
        \begin{tabular}{|c|c|c|l|}
            \hline
            \textbf{Type} & \textbf{Alias} & \textbf{Description} \\
            \hline
            \texttt{bool} & & A truth value -- either \texttt{true} or \texttt{false} \\
            \hline
            \texttt{u8} & \texttt{byte} & Unsigned 8-bit integer \\
            \texttt{i8} & & Signed 8-bit integer \\
            \hline
            \texttt{u16} & & Unsigned 16-bit integer \\
            \texttt{i16} & & Signed 16-bit integer \\
            \hline
            \texttt{u32} & & Unsigned 32-bit integer \\
            \texttt{i32} & \texttt{int} & Signed 32-bit integer \\
            \hline
            \texttt{u64} & & Unsigned 64-bit integer \\
            \texttt{i64} & \texttt{long} & Signed 64-bit integer \\
            \hline
            \texttt{f32} & \texttt{float} & 32-bit floating-point number \\
            \texttt{f64} & \texttt{double} & 64-bit floating-point number \\
            \hline
        \end{tabular}
        \caption{Primitive Types}\label{tab:primitive-types}
    \end{table}

    % TODO type hierarchy

    These types have the following type hierarchy:
    \begin{itemize}
        \item \texttt{int8 < int16 < int32 < int64}
        \item \texttt{uint8 < uint16 < uint32 < uint64}
        \item \texttt{uint\(n\) < int\((n+k)\)}
        \item \texttt{float32 < float64}
        \item \texttt{[u]int\(n\) < float64}
        \item \texttt{[u]int\(n\) < float32} where \(n < 64\).
    \end{itemize}

    Note that \texttt{bool} is not related to anything.
    Unlike in other language, \texttt{bool} is treated as a distinct type with abstract true/false values, rather than a synonym for integers 1/0, respectively.

    \subsection{The Unit Type}

    Ths unit type, denotes by empty brackets `\texttt{()}', is a special type meaning \textit{empty}.
    Representing an object with no value, it is a zero-width type, meaning a symbol of this type occupies no space.
    For examples of use, a unit is returned from a block which returns no value.

    \begin{lstlisting}[language=CustomLang]
let a = {}; // a: ()
    \end{lstlisting}

    Note that a unit is neither an r-- nor an l-value, so it cannot be used in expressions.
    That is, the below program errors before even trying to find a candidate for \texttt{+}.

    \begin{lstlisting}[language=CustomLang]
() + 1; // error: expected rvalue, got ()
    \end{lstlisting}

    \subsection{Pointer Types}\label{subsec:pointer-types}

    The use of pointers is designed to be minimal in Edel, but they are available nonetheless.
    Simply put, a pointer stores the \textit{location} (or \textit{address}) of a variable, which can then be used to later reference and access that data.
    A pointer is indicated by a star `\texttt{*}' preceding the type;
    that is, type \texttt{*T} is telling the compiler \textit{hey, see this location? Data of type} \texttt{T} \textit{is stored there}.

    Subsection ~\ref{subsec:address-of-&-dereferencing} explains the use of the \texttt{\&} and \texttt{*} unary operators to manipulate pointers.

    Pointers do not fit into the type hierarchy, and cannot be implicitly cast.
    For safety, it is best not to cast pointer types unless one really knows and understands what is going on.

    \subsection{Function Types}\label{subsec:function-types}

    % TODO

    \subsection{Casting}\label{subsec:type-casting}

    Casting refers to the changing of a type, and can be split into two categories: implicit, and explicit.
    Note that the destination type of either cast \textbf{cannot} be zero-sized.

    Implicit casting is where a cast is done behind-the-scenes.
    Type \(a\) may be implicitly cast (also known as \textit{coerced}) to another type \(b\) if and only if \(a < b\), that is, \(a\) is a subtype of \(b\).
    In other words, \(a\) can be coerced into \(b\) if \(b\) is sufficient to represent \(a\) without data loss.
    For example, \texttt{u8} may be coerced into \texttt{u64}, or \texttt{float} into \texttt{double}, but \textbf{not} \texttt{i32} into \texttt{i16}.
    Implicit casting is done automatically in the following:

    \begin{lstlisting}[language=CustomLang]
let a: double = 3.14; // float -> double
    \end{lstlisting}

    On the other hand, explicit casting violates these rules, and hence the cast may be \textit{explicit}.
    Note that not all datatypes may be cast between each-other using explicit casting, but the subtype relationships may be largely violated.

    \begin{lstlisting}[language=CustomLang]
let a: u8 = 42; // error!
let a = 42 as u8; // i32 -> u8
    \end{lstlisting}

    \section{Functions}\label{sec:functions}

    Functions are labelled blocks of code which may be called by the programmer.
    A function may take some input values, known as \textit{arguments}, and may optionally return a value.

    An example of a function which takes two integers and returns an integer:
    \begin{lstlisting}[language=CustomLang]
func add(a: int, b: int) -> int {
    return a + b;
}
    \end{lstlisting}

    That is, a function is defined using the \texttt{func} keyword, followed by a name.
    Parameters in the form \texttt{name: type} are listed, comma-separated, between brackets.
    Note that if no arguments are required, the brackets may be omitted altogether.
    The function's return type is listed after an arrow; if no return type is needed, this may be omitted.
    For example,
    \begin{lstlisting}[language=CustomLang]
func foo {
    // ...
}

func bar(a: float) { // equivalent to `-> ()'
    // ...
}
    \end{lstlisting}

    \subsection{Calling Functions}

    Functions are called by following the function name with brackets `\texttt{()}' (this is the operator name).
    Any arguments to be supplied to the function are comma-separated inside these brackets, known as an \textit{argument list}.
    A function with a matching signature is then found (see subsection ~\ref{subsec:function-overloading} on function overloading), and its return value produced.

    \begin{lstlisting}[language=CustomLang]
func add(a: int, b: int) -> int {
    return a + b;
}
let result = add(1, 4); // result: int = 5
    \end{lstlisting}

    Note that \texttt{()} is an operator like any other.
    Therefore, if the subject is not a function type, an overload for \texttt{operator()} is searched for.
    The signature will be the subject type followed by each argument's type.
    For example,

    \begin{lstlisting}[language=CustomLang]
5(3.14, true); // operator()(i32, f32, bool)
    \end{lstlisting}

    \subsection{Returning a Value}

    The \texttt{return} keyword is used to exit/return from the current function.
    Optionally, the keyword may be followed by an expression, in which case that value will be returned.
    The type of the returned expression must match, or be a subtype of, the function's return type.
    Furthermore, if the function returns a non-unit type, a return statement must be present.

    \begin{lstlisting}[language=CustomLang]
func foo {
    return 42; // error: cannot convert i32 to ()
}

func bar -> int {
} // error: missing return statement
    \end{lstlisting}

    Note that \textit{all} code paths should return, even if you, the programmer, knows that a path is never taken.
    % TODO update if constant optimisation added
    That is because the compiler cannot currently recognise, for example, always-falsy conditions.

    \begin{lstlisting}[language=CustomLang]
func foo -> int {
    if false {
    } else {
        return 42;
    }
} // error: missing return statement
    \end{lstlisting}

    \subsubsection{Unreachable Code}

    As \texttt{return} exits the function, any code after said statement is \textit{unreachable}.
    This code will never be executed, and a warning will be reported.

    \begin{lstlisting}[language=CustomLang]
func foo {
    return;
    a; // warning: unreachable code detected
}
    \end{lstlisting}

    The above code would usually result in an error, as symbol `\texttt{a}' is not defined.
    However, as the code is not reached, no error is raised.

    \subsection{Function Declaration}\label{subsec:function-declaration}

    A function declaration is one which has no body.
    For example,

    \begin{lstlisting}[language=CustomLang]
func add(a: int, b: int) -> int;
    \end{lstlisting}

    As functions in Edel do not require forward-declaration, what does this do?
    Depending on the compiler flag \texttt{--function-placeholder}, it does one of two things.

    \paragraph*{Enforce Existence}

    The default, \texttt{--no-function-placeholder}, a function declaration is used to inform the programmer about the existence of a function.
    One common use is in auto-generated standard library files, where the function exists but is implemented internally.
    If the function does not exist, an error will be raised.

    \begin{lstlisting}[language=CustomLang]
func imaginary_function(x: int) -> bool; // error: function was declared but does not exist
    \end{lstlisting}

    \paragraph*{Generate Placeholder}

    In this instance, if a function is declared but does not exist, the function will be created to return zero, `\texttt{()}', false, or null, depending on the return type.
    This is useful when a function is yet to be implemented.

    For example, the following are identical.
    \begin{lstlisting}[language=CustomLang]
func imaginary_function(x: int) -> bool;
// ==
func imaginary_function(x: int) -> bool {
    return false;
}
    \end{lstlisting}


    Note that as this generates a definition, it cannot be re-defined.

    \subsection{Function Overloading}\label{subsec:function-overloading}

    A function in Edel is not only identified by its name, but also its \textit{signature}.
    The signature of a function is a combination of its name and parameter types.
    In an Edel program, a function signature must be unique, or the result would be an error.
    This means that multiple functions with the same name can exist.
    This is known as function overloading, and a variant of a function with the same name but different parameters is called an \textit{overload}.

    \begin{lstlisting}[language=CustomLang]
func add(a: int, b: int) -> int {
    return a + b;
}
func add(a: float, b: float) -> float {
    return a + b;
}
    \end{lstlisting}

    In this case, a reference to \texttt{add} no longer just refers to one symbol.
    Instead, it becomes what is known as an \textit{overload set}, and more information is required for a resolution.

    \begin{lstlisting}[language=CustomLang]
add; // error: unable to resolve symbol `add'
    \end{lstlisting}

    \subsubsection{Overload Resolution}

    This is the process by which an overload is selected.
    Firstly, a list of candidates is generated; this would be all functions with a matching name.
    A candidate is considered a match if the number of arguments match and the type of each parameter matches.
    Due to subtyping rules, the latter check may lead to the generating of multiple `matching' candidates.
    In such a case, the candidate with the highest number of identical types wins.

    \begin{lstlisting}[language=CustomLang]
func add(a: int, b: int) -> int;
func add(a: long, b: long) -> long;

add(1, 2); // signature is add(int, int)
    \end{lstlisting}

    In the above snippet, both candidates match the callee.
    However, the former wins as \texttt{int == int} while \texttt{int < long}.

    If there are still ties at this point, an error is raised as a candidate cannot be decided.
    In this case, it is best to explicitly cast one or more arguments to reach the desired candidate.

    \begin{lstlisting}[language=CustomLang]
func add(a: int, b: long) -> long;
func add(a: long, b: int) -> long;

add(1, 2); // error: unable to resolve
    \end{lstlisting}

    \section{Operators}

    Operators are special symbols which combine one or two expressions, known as unary and binary operators, respectively.
    The table below lists the built-in operators alongside their behaviour and associativity.
    Operators start from tightest to loosest precedence, with those of the same precedence grouped.

    \begin{table}[h]
        \centering
        \caption{Built-In Operators}
        \begin{tabular}{|l|l|c|c|}
            \hline
            \textbf{Operator} & \textbf{Description} & \textbf{Associativity} & \textbf{Overloadable} \\
            \hline
            \texttt{a.b} & Member access & \(\longrightarrow\) & No \\
            \hline
            \texttt{a()} & Function call & \(\longrightarrow\) & Yes \\
            \hline
            \texttt{(T) a} & Primitive cast & \multirow{4}{*}{\(\longleftarrow\)} & \multirow{4}{*}{No} \\
            \texttt{\&a} & Address-of & & \\
            \texttt{*a} & Dereference & & \\
            \texttt{registerof a} & Register lookup & & \\
            \hline
            \texttt{-a} & Negation & \multirow{3}{*}{\(\longleftarrow\)} & \multirow{3}{*}{Yes} \\
            \texttt{~a} & Bitwise NOT & & \\
            \texttt{!a} & Boolean NOT & & \\
            \hline
            \texttt{a as T} & Cast & \(\longleftarrow\) & No \\
            \hline
            \texttt{a * b} & Multiplication & \multirow{3}{*}{\(\longrightarrow\)} & \multirow{3}{*}{Yes} \\
            \texttt{a / b} & Division & & \\
            \texttt{a \% b} & Modulo (remainder) & & \\
            \hline
            \texttt{a + b} & Addition & \multirow{2}{*}{\(\longrightarrow\)} & \multirow{2}{*}{Yes} \\
            \texttt{a - b} & Subtraction & & \\
            \hline
            \texttt{a << b} & Bitwise left shift & \multirow{2}{*}{\(\longrightarrow\)} & \multirow{2}{*}{Yes} \\
            \texttt{a >> b} & Bitwise left shift & & \\
            \hline
            \texttt{a == b} & \multirow{6}{*}{Relational operators} & \multirow{6}{*}{\(\longrightarrow\)} & \multirow{6}{*}{Yes} \\
            \texttt{a != b} & & & \\
            \texttt{a < b} & & & \\
            \texttt{a <= b} & & & \\
            \texttt{a > b} & & & \\
            \texttt{a >= b} & & & \\
            \hline
            \texttt{a \& b} & Bitwise AND & \multirow{1}{*}{\(\longrightarrow\)} & Yes \\
            \hline
            \texttt{a \string^ b} & Bitwise XOR & \multirow{1}{*}{\(\longrightarrow\)} & Yes \\
            \hline
            \texttt{a | b} & Bitwise OR & \multirow{1}{*}{\(\longrightarrow\)} & Yes \\
            \hline
            \texttt{a = b} & Assignment & \multirow{1}{*}{\(\longleftarrow\)} & No \\
            \hline
        \end{tabular}\label{tab:builtin-operators}
    \end{table}

    \subsection{Casting}\label{subsec:op-casting}

    The concept of casting is explained in subsection ~\ref{subsec:type-casting}.
    Note that the table above presents two casting operators: \texttt{(T) a} and \texttt{a as T}.

    The former is a \textit{primitive} cast, meaning it only supports casting to primitive, atomic types.
    This cast is direct and is ignorant of the type of \texttt{a} and whether the cast is meaningful.

    The latter, using the \texttt{as} keyword, is a straightforward explicit cast.
    There are no restrictions on \texttt{T} and, as explained in subsection ~\ref{subsec:type-casting}, this cast still must obey some rules.

    \begin{lstlisting}[language=CustomLang]
let a = (int) 3.14; // ok
let b = (*int) 0; // error
let c = 0 as *int; // ok
    \end{lstlisting}

    Also note the difference in precedence: the \texttt{as} cast is slightly looser than other unary operators.

    \begin{lstlisting}[language=CustomLang]
!1 as bool; // same as `(!1) as bool' NOT `!(1 as bool)'
    \end{lstlisting}

    \subsection{Address-Of \& Dereferencing}\label{subsec:address-of-&-dereferencing}

    Both operations involve pointers, with the former generating pointers, and the latter `consuming' them.

    The address-of operator, \texttt{\&}, calculates the memory address of a symbol.
    That is, if \texttt{a: T}, then \texttt{\&a} is of type \texttt{*T}.

    Dereferencing, then, is address-of's twin; it retrieved the value stored at a pointer.
    That is, if \texttt{p: *T}, then \texttt{*a} is of type \texttt{T}.

    \begin{lstlisting}[language=CustomLang]
let a: int = 5;
let p: *int = &a;
let b: int = *p; // then a == b
    \end{lstlisting}

    \subsection{Register Lookup}

    This is a lower-level operator meant for exploration of the processor's registers.

    Expecting a symbol argument, the \texttt{registerof} operator returns the index (offset) of where the given symbol is located.
    This allows one to gain insight into the register allocation algorithm.

    For example, given the basic program below, one would expect the following.

    \begin{lstlisting}[language=CustomLang]
let a = 5;
registerof a; // = 11, corresponding to $r1
    \end{lstlisting}

    If the symbol has not been loaded into registers (i.e., it has not been used at all/recently), the operator returns \(-1\).

    \begin{lstlisting}[language=CustomLang]
let a: int;
registerof a; // = -1
    \end{lstlisting}

    \section{Control Flow}

    \section{Namespaces}

    Namespaces allow the programmer to group related functions, variables etc. together in an environment.
    Not only does this aid in organisation, but also helps to prevent naming collisions.

    A namespace is created using the \texttt{namespace} keyword, and may contain any structure which is permitted at the global level.

    \begin{lstlisting}[language=CustomLang]
namespace maths {
    func cos(x: double) -> double;
}
    \end{lstlisting}

    To reference a name outside the namespace, its name must be prefixed by the namespace's name.

    \begin{lstlisting}[language=CustomLang]
cos(0); // error: unable to resolve symbol `sin'
maths.cos(0); // = 1
    \end{lstlisting}

    \section{Modules \& Imports}

    \section{Standard Library}

\end{document}
