\documentclass[10pt]{article}
\usepackage[margin=0.75in]{geometry}
\usepackage{makecell}
\usepackage{longtable}
\usepackage{amsmath, amssymb}
\usepackage{graphicx} % Required for inserting images

\setlength{\parindent}{0pt}

\title{Processor Documentation}
\author{Ruben Saunders}
\date{August 2024}

\begin{document}

\maketitle

\section{Principals}

\begin{itemize}
    \item This processor will operate a RISC instruction set.
    \item This processor has a word size of 64 bits, and supports both floats (4 bytes) and doubles (8 bytes).
    \item The instruction set will provide methods to load values into and out of registers.
    Then, most operations will be on registers.
    \item Load/store instructions operate on 32-bit immediates.
    \item Arithmetic and logic instructions operate on full registers, so 64-bit.
\end{itemize}

\section{Registers}

See below for a list of registers.
All registers are 64-bit.
Register names are preceded by a dollar `\$' sign.

\bigskip
\begin{longtable}{|c|l|l|l|}
    \hline
    \textbf{Symbol} & \textbf{Name} & \textbf{Bit} & \textbf{Description} \\
    \hline
    \multicolumn{4}{|c|}{\textbf{Special Registers}} \\
    \hline
    \$ip & Instruction Pointer &  & Point to next address to execute as an instruction. \\
    \hline
    \$sp & Stack Pointer &  & Top address of the stack. \\
    \hline
    \$fp & Frame Pointer &  & Point to the next byte beyond the last stack frame. \\
    \hline
    \$flag & Flag Register & 8-64 & \\
    \cline{3-4}
    & & 5-7 & \makecell[l]{Error flag.\\%
    \(\bullet\;\) 000: no error.\\%
    \(\bullet\;\) 001: invalid opcode, opcode in \$ret.\\%
    \(\bullet\;\) 010: segfault, address in \$ret.\\%
    \(\bullet\;\) 011: register segfault, register offset in \$ret.\\%
    \(\bullet\;\) 100: invalid syscall, opcode in \$ret.\\%
    } \\
    \cline{3-4}
    & & 4 & \makecell[l]{Execution status: 1=executing, 0=halted.\\%
    Can be used to halt the processor.} \\
    \cline{3-4}
    & & 3 & \makecell[l]{Zero flag.\\%
    Indicates if register is zero.\\%
    Updated on most instructions' dest register.} \\
    \cline{3-4}
    & & 0-2 & \makecell[l]{Comparison bits.\\%
    \(\bullet\;\) 000: not equal.\\%
    \(\bullet\;\) 001: equal.\\%
    \(\bullet\;\) 010: less than.\\%
    \(\bullet\;\) 011: less than or equal to.\\%
    \(\bullet\;\) 110: greater than.\\%
    \(\bullet\;\) 111: greater than or equal to.\\%
    } \\
    \hline
    \$ret & Return Value Register & & \makecell[l]{Contains value returned from function, syscall, etc.\\%
    Contains process exit code on halt.} \\
    \hline
    \$zero & Zero & & Hardwired to zero. \\
    \hline \hline
    \multicolumn{4}{|c|}{\textbf{General Purpose Registers}} \\
    \hline
    \$r1 -- \$r16 & GPRs &  & Register for general use. \\
    \hline
    \$s1 -- \$s8 & Preserved GPRs &  & \makecell[l]{Register for general use.\\Values are preserved in stack frame.} \\
    \hline
\end{longtable}

\section{Addressing Modes}

An argument can have the following type.

\medskip
\begin{tabular}{|c|l|l|l|l|}
    \hline
    \textbf{Indicator} & \textbf{Name} & \textbf{Syntax} & \textbf{Operation} & \textbf{Size} \\
    \hline
    00 & Immediate & \texttt{imm} & \texttt{imm} & 32 \\
    01 & Memory & \texttt{(mem)} & \texttt{Mem[mem]} & 32\\
    10 & Register & \texttt{\$reg} & \texttt{Reg[\$reg]} & 8 \\ 
    11 & Register Indirect & \texttt{n(\$reg)} & \texttt{Mem[Reg[\$reg] + n]} & \texttt{\$reg}=8, \texttt{\$n}=24 \\
    \hline
\end{tabular}

\bigskip
The following values are used on the ISA specification:
\medskip

\begin{tabular}{|c|c|l|}
    \hline
    \textbf{Argument} & \textbf{Size} & \textbf{Comment} \\
    \hline
    \texttt{<reg>} & 8 & Register offset. \\
    \hline
    \texttt{<value>} & 2 + 32 & \makecell[l]{Any listed addressing mode.\\%
    2 indicator bits, 32 for data.} \\
    \hline
    \texttt{<addr>} & 2 + 32 & \makecell[l]{Any listed addressing mode \textbf{except} immediate.\\%
    2 indicator bits, 32 for data.} \\
    \hline
\end{tabular}

\section{Instruction Set}

\textbf{Note} the \(\square\) symbol means that this instruction does not take a conditional test suffix.

\textbf{Note} that mnemonics support overloading.
That is, the same mnemonic can have many argument signatures.
Optional arguments are listed using square brackets \texttt{[optional]} versus mandatory arguments \texttt{<mandatory>}.

\textbf{Note} for all arithmetic and logical instructions with signature \texttt{<reg> <reg> <value>}, the first register is optional.
If omitted, the supplied register is duplicated.
That is, \texttt{\$r, \$v} becomes \texttt{\$r, \$r, \$v}.

\textbf{Note} all arithmetic operations and the compare operation take a datatype.

\begin{longtable}{|c|l|l|}
    \hline
    \textbf{Instruction} & \textbf{Syntax} & \textbf{Operation/Comments} \\
    \hline
    \multicolumn{3}{|c|}{\textbf{Data Transfer}} \\
    \hline
    Load & \texttt{load <reg> <value>} & \makecell[l]{Load a half-word (32-bit) into a register.\\\texttt{Reg[\$reg] = \$value}} \\
    \hline
    Load Upper & \texttt{loadu <reg> <value>} & \makecell[l]{Load a half-word (32-bit) into the upper half of a register.\\\texttt{Reg[\$reg][32:] = \$value}} \\
    \hline
    Load Word & \texttt{loadw <reg> <value>} & \makecell[l]{\textit{Pseudo-instruction.}\\%
    Loads a word (64-bit) into a register.\\%
    \texttt{load \$reg \$value[:32]}\\%
    \texttt{loadu \$reg \$value[32:]}} \\
    \hline
    Zero & \texttt{zero <reg>} & \makecell[l]{\textit{Pseudo-instruction.}\\%
    Zeroes/clears a register.\\%
    \texttt{xor \$reg, \$reg}} \\
    \hline
    Store & \texttt{store <reg> <addr>} & \makecell[l]{Copy from register to memory.\\\texttt{Mem[\$addr] = Reg[\$reg]}} \\
    \hline \hline
    \multicolumn{3}{|c|}{\makecell[c]{\textbf{Arithmetic}\\%
    All arithmmetic operations, bar \texttt{mod}, expect a datatype.}} \\
    \hline
    Add & \texttt{add <reg> <reg> <value>} & \makecell[l]{Add value to a register.\\%
    \texttt{Reg[\$reg1] = Reg[\$reg2] + \$value}} \\
    \hline
    Subtract & \texttt{sub <reg> <reg> <value>} & \makecell[l]{Subtract value from a register.\\%
    \texttt{Reg[\$reg1] = Reg[\$reg2] - \$value}} \\
    \hline
    Multiply & \texttt{mul <reg> <reg> <value>} & \makecell[l]{Multiply register by a value.\\%
    \texttt{Reg[\$reg1] = Reg[\$reg2] \(\times\) \$value}} \\
    \hline
    Division & \texttt{div <reg> <reg> <value>} & \makecell[l]{Divide a register by a value, store as double.\\%
    \texttt{Reg[\$reg1] = Reg[\$reg2] \(\div\) \$value}} \\
    \hline
    Modulo & \texttt{mod <reg> <reg> <value>} & \makecell[l]{Calculate the remainder when dividing a register by a value.\\%
    The register is treated as a signed word,\\%
    the value as a signed half-word.\\%
    \texttt{Reg[\$reg1] = Reg[\$reg2] \(\mod\) \$value}} \\
    \hline \hline
    \multicolumn{3}{|c|}{\textbf{Branching}} \\
    \hline
    Compare & \texttt{cmp <reg> <value>} & \makecell[l]{Compare \texttt{\$1} with \texttt{\$2}, setting comparison bits in flag register.\\%
    E.g., set \texttt{lt} iff \texttt{\$1 < \$2}.\\%
    \textbf{Note} Z flag is set depending on value, not register.} \\
    \hline
    Branch & \texttt{b<cnd> <value>} & \makecell[l]{\textit{Pseudo-instruction}\\%
    Branch to the given address if comparison matches conditional.\\%
    \texttt{load<cnd> \$ip, \$value}} \\
    \hline
    Jump \(\square\) & \texttt{jmp <value>} & \makecell[l]{\textit{Pseudo-instruction.}\\%
    \texttt{load \$ip \$value}} \\
    \hline \hline
    \multicolumn{3}{|c|}{\textbf{Logical}} \\
    \hline
    Not & \texttt{not <reg> <reg>} & \makecell[l]{Bitwise NOT a register.\\%
    \texttt{Reg[\$reg1] = \(\sim\) Reg[\$reg2]}} \\
    \hline
    And & \texttt{and <reg> <reg> <value>} & \makecell[l]{Bitwise AND between register and value.\\%
    \texttt{Reg[\$reg1] = Reg[\$reg2] \& \$value}} \\
    \hline
    Or & \texttt{or <reg> <reg> <value>} & \makecell[l]{Bitwise OR between register and value.\\%
    \texttt{Reg[\$reg1] = Reg[\$reg2] | \$value}} \\
    \hline
    Exclusive Or & \texttt{xor <reg> <reg> <value>} & \makecell[l]{Bitwise exclusive-OR between register and value.\\%
    \texttt{Reg[\$reg1] = Reg[\$reg2] \(\oplus\) \$value}} \\
    \hline
    Right Shift & \texttt{shr <reg> <reg> <value>} & \makecell[l]{Logically shift the register right an amount.\\%
    \texttt{Reg[\$reg1] = Reg[\$reg2] \(\gg\) \$value}} \\
    \hline
    Left Shift & \texttt{shl <reg> <reg> <value>} & \makecell[l]{Logically shift the register left an amount.\\%
    \texttt{Reg[\$reg1] = Reg[\$reg2] \(\ll\) \$value}} \\
    \hline \hline
    \multicolumn{3}{|c|}{\textbf{Stack}} \\
    \hline
    Push & \texttt{push <value>} & \makecell[l]{\textit{Pseudo-instruction}\\%
    Push a 32-bit value onto the stack.\\%
    \texttt{sub \$sp, 8}\\%
    \texttt{loadu \$r1, <value>}\\%
    \texttt{store \$r1, (\$sp)}\\%
    \texttt{add \$sp, 4}\\%
    \textbf{Note} for efficiency, this is implemented as an instruction.} \\
    \hline
    Push Word & \texttt{pushw <value>} & \makecell[l]{\textit{Pseudo-instruction}\\%
    Push a 64-bit word onto the stack.\\%
    \texttt{sub \$sp, 8}\\%
    \texttt{loadw \$r1, <value>}\\%
    \texttt{store \$r1, (\$sp)}} \\
    \hline
    Pop & \multicolumn{2}{l|}{\makecell[l]{The pop operation is not implemented due to its simplistic nature.\\%
    I.e., to pop a word from the stack:\\%
    \texttt{sub \$sp, 8}\\%
    And to store it in a register:\\%
    \texttt{load \$r1, (\$sp)}}} \\
    \hline \hline
    \multicolumn{3}{|c|}{\textbf{Functions}} \\
    \hline
    Function Call & \texttt{call <value>} & \makecell[l]{Call procedure at location \texttt{value}.\\%
    More complex than \texttt{load ip, \$value} as pushes stack frame.} \\
    \hline
    Return & \texttt{ret} & \makecell[l]{Return from function call.\\%
    Restores key registers (undoes \texttt{call}).} \\
    \hline
    System Call & \texttt{syscall <value>} & \makecell[l]{Invoke the system call mapped to the given value.\\%
    See the respective section for mappings.} \\
    \hline \hline
    \multicolumn{3}{|c|}{\textbf{Miscellaneous}} \\
    \hline
    No-Operation \(\square\) & \texttt{nop} & \makecell[l]{Useless operation; do nothing.\\%
    Equivalent to \texttt{or r1, 0}.\\%
    \textbf{Note} For efficiency, implemented as instruction.} \\
    \hline
    Exit & \texttt{exit [value]} & \makecell[l]{\textit{Pseudo-instruction}\\%
    Exit the program, optionally with an exit code in \$ret.\\%
    If code provided: \texttt{load \$ret, <value>}\\%
    \texttt{syscall <opcode: exit>}} \\
    \hline
\end{longtable}

\subsection{Pseudo-Instructions}

These are instructions which are not necessary for full functionality, but are provided for usefulness.
They may be implemented using other instructions.
It is up to the implementer whether to implement these as actual instructions or expand them to their equivalent form.

\subsection{Instruction Layout}

All instructions are encoded in a single 64-bit word.
The layouts of various types is listed below.
The size field stated the size in bits of this field.
From top-to-bottom, the table starts at the least-significant bit.

\textbf{Note}, the opcode of each instruction is not decided upon; it may be any value as long as the instruction set is implemented.
The only exception is \texttt{nop}, which maps to a fully-zeroed word.

\paragraph{Generic Layout}
This outlines the generic structure of an instruction.
The first section of the table refers to the `header'.

\bigskip
\begin{tabular}{|c|l|l|}
    \hline
    \textbf{Bit} & \textbf{Purpose} & \textbf{Comments} \\
    \hline
    0-5 & Opcode & \\
    \hline
    6-9 & Conditional test & \makecell[l]{These bits are tested against \$flag to determine if instruction is executed or skipped.\\%
    \(\bullet\;\) 1111: skip test.\\%
    \(\bullet\;\) 1001: test if zero flag is set.\\%
    \(\bullet\;\) 1000: test if zero flag is unset.\\%
    \(\bullet\;\) Otherwise: match lower 3 bits to \$flag.\\%
    \textbf{Note} these bits are omitted if \(\square\) is shown.} \\
    \hline
    \hline
    10-64 & \multicolumn{2}{l|}{Instruction dependant.} \\
    \hline
\end{tabular}

\paragraph{Data-Type Indicator}
Some instructions have a field to specify the data-type of the data being operated on.
These bits are after the ordinary header, and are as follows:

\bigskip
\begin{tabular}{|c|c|c||c|l|}
    \hline
    \makecell[c]{\textbf{Bit 0}\\Decimal?} & \makecell[c]{\textbf{Bit 1}\\Signed?} & \makecell[c]{\textbf{Bit 0}\\Full or half word?} & \textbf{Suffix} & \textbf{Comments} \\
    \hline
    0 & 0 & 0 & hu & 32-bit unsigned integer. \\
    \hline
    0 & 0 & 1 & [u] & 64-bit unsigned integer. \\
    \hline
    0 & 1 & 0 & hi & 32-bit signed integer. \\
    \hline
    0 & 1 & 1 & i & 64-bit signed integer. \\
    \hline
    1 & 0 & 0 & f & 32-bit float. \\
    \hline
    1 & 0 & 1 & d & 64-bit double. \\
    \hline
\end{tabular}

\bigskip
Datatypes may be interpreted slightly differently, depending on the instruction.

\begin{itemize}
    \item Arithmetic operations: the datatype refers to the type of the first data to be operated on.
    The last argument is always considered a 32-bit signed integer or float.
    That is, in \texttt{add.u \$r1, -75}, \texttt{\$r1} is assumed to hold an unsigned 64-bit integer, but \texttt{-75} is a 32-bit signed integer, while the result also be an unsigned 64-bit integer.

\end{itemize}

\section{Calling Convention}

Despite being a RISC processor, this processor will support explicit \texttt{call} and \texttt{ret} functions which will aid in pushing and popping a stack frame.
For ease of programming, multiple actions are taken in each to maintain structure, so they are not pseudo-instructions.

\subsection{Function Invocation}

To call a function [at] \texttt{func} with \(n\) arguments:

\medskip
\texttt{%
push <arg1>\\%
...\\%
push <arg\(n\)>\\%
push \(n\)\\%
call <func>
}

\textbf{Note} when zero arguments are needed, still \texttt{push 0} to indicate this.
\medskip

\begin{tabular}{|r l||r l|}
     \hline
     \multicolumn{4}{|c|}{\textbf{Stack}} \\
     \hline
     \multicolumn{2}{|c||}{\textbf{Before}} & \multicolumn{2}{c|}{\textbf{After}} \\
     \hline
     & & preserved GP registers & \(\leftarrow\) \$sp \\
     & & old ip & \\
     & & old fp & \(\leftarrow\) \$fp \\
     & & \(n\) & \\
     & & args & \\
     \texttt{xxx} & \(\leftarrow\) \$sp & \texttt{xxx} & \\
     \hline
\end{tabular}

\subsection{Function Returning}

To return from the function invoked in the previous sub-section, we need only a call to \texttt{ret}.
This will restore and pop the stack frame, as well as handle any arguments the user pushed.
The following operations take place:

\texttt{%
Reg[\$ip] = old ip\\%
Reg[\$fp] = old fp\\%
Reg[\$sp] = loc(xxx)\\%
}

\subsection{Argument Retrieval}

The frame pointer points to the top of the previous frame.
Using the diagram above, it is possible to retrieve an argument from the stack.
It is important to note that the size of the additional information pushed via the processor may theoretically vary, and so referencing and relying on knowledge of this size is unadvised.

\begin{center}
    \(i\): argument index, 0-indexed; \(n\): number of arguments.
    
    \texttt{Arg \(i\) = Reg[\$fp] - 4 * (1 + \(n\) - \(i\))}
\end{center}

\section{System Call}

System calls are core functionality abstracted inside the processor.
Actions are assigned operation codes and invoked via \texttt{syscall <opcode>}.
Optionally, each read arguments from general-purpose registers \texttt{r1} onward.

\bigskip
\begin{longtable}{|c|c|l|l|l|}
    \hline
    \textbf{Service} & \textbf{Opcode} & \textbf{Arguments} & \textbf{Operation} & \textbf{Result} \\
    \hline
    \multicolumn{5}{|c|}{\textbf{Output}} \\
    \hline
    print\_int & 1 & \texttt{\$r1} = integer & Print 64-bit integer. & \textit{None} \\
    \hline
    print\_float & 2 & \texttt{\$r1} = float & Print 32-bit float. & \textit{None} \\
    \hline
    print\_double & 3 & \texttt{\$r1} = double & Print 64-bit double. & \textit{None} \\
    \hline
    print\_char & 4 & \texttt{\$r1} = byte & Print byte as ASCII character. & \textit{None} \\
    \hline
    print\_string & 5 & \texttt{\$r1} = string address & Print null-terminated string at the address. & \textit{None} \\
    \hline \hline
    \multicolumn{5}{|c|}{\textbf{Input}} \\
    \hline
    read\_int & 6 & \textit{None} & Read a signed 64-bit integer. & \texttt{\$ret} = integer \\
    \hline
    read\_float & 7 & \textit{None} & Read a 32-bit float. & \texttt{\$ret} = float \\
    \hline
    read\_double & 8 & \textit{None} & Read a 64-bit double. & \texttt{\$ret} = double \\
    \hline
    read\_char & 9 & \textit{None} & Read an ASCII character. & \texttt{\$ret} = character \\
    \hline
    read\_string & 10 & \makecell[l]{\texttt{\$r1} = string address\\%
    \texttt{\$r2} = max length} & \makecell[l]{Read a null-terminated string into given address.\\%
    String is truncated to maximum length.} & \textit{None} \\
    \hline \hline
    \multicolumn{5}{|c|}{\textbf{Program Flow}} \\
    \hline
    exit & 11 & \textit{None} & \makecell[l]{Exit program.\\%
    \textbf{Note} process exit code is located in \$ret.} & \textit{None} \\
    \hline \hline
    \multicolumn{5}{|c|}{\textbf{Debug}} \\
    \hline
    print\_regs & 100 & \textit{None} & Print hexadecimal value of each register. & \textit{None} \\
    \hline
    print\_mem & 101 & \makecell[l]{\texttt{\$r1} = start address\\%
    \texttt{\$r2} = segment length} & Print hexadecimal bytes of memory segment. & \textit{None} \\
    \hline
    print\_stack & 102 & \textit{None} & Print bytes of the stack. & \textit{None} \\
    \hline
\end{longtable}

\end{document}
