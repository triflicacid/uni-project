\documentclass{article}
\usepackage{longtable}
\usepackage{multirow}
\usepackage{makecell}
\usepackage{enumitem}
\usepackage{graphicx}
\usepackage[T1]{fontenc}

\newlength{\drop}

\begin{document}

\begin{titlepage}
    \drop=0.1\textheight
    \centering
    \vspace*{\baselineskip}
    \rule{\textwidth}{1.6pt}\vspace*{-\baselineskip}\vspace*{2pt}
    \rule{\textwidth}{0.4pt}\\[\baselineskip]
    {\LARGE PROJECT REQUIREMENTS}\\[0.2\baselineskip]
    \rule{\textwidth}{0.4pt}\vspace*{-\baselineskip}\vspace{3.2pt}
    \rule{\textwidth}{1.6pt}\\[\baselineskip]
    \scshape
    CS310 Computer Science Project\\
    SUPERVISOR: GIHAN MUDALIGE\par
    \vspace*{2\baselineskip}
    \includegraphics[scale = 0.45]{assets/WarwickCrest.pdf}\\
    \vspace*{\baselineskip}
    {\itshape University of Warwick \\ Department of Computer Science\par}
    \vfill
    {\scshape AUGUST 2024} \\
    {\large RUBEN SAUNDERS}\par
\end{titlepage}

\newpage

\section{The Processor}

Core to the project, a RISC-based processor will be designed and an emulator written.
The emulator is but a program which will implement the processor specification.

\subsection{Emulator Requirements}

\begin{quote}
    ``The emulator must be able to accept a binary file, read it, and execute the contents according to a RISC processor specification.''
\end{quote}

\begin{enumerate}[label=R\arabic*]
    \item The emulator shall be able to accept, read, and execute a binary.
    \begin{enumerate}[label=\theenumi.\arabic*]
        \item The emulator shall be able to accept a file name as an argument.
        \item The emulator shall be able to open and read a file in binary mode.
    \end{enumerate}
    \item The emulator shall execute a binary according to a pre-defined RISC processor specification.
    \begin{enumerate}[label=\theenumi.\arabic*]
        \item The emulator shall implement a processor specification.
        \item The specification shall define a RISC-based processor.
        \begin{enumerate}[label=\theenumii.\arabic*]
            \item The specification shall fully outline the ISA.
            \item The specification shall fully outline expected memory structure.
            \item The specification shall fully outline all available registers.
        \end{enumerate}
    \end{enumerate}
    \item The emulator shall be able to indicate its past and current state.
    \begin{enumerate}[label=\theenumi.\arabic*]
        \item The emulator shall be able to print to the console.
        \item The emulator shall be able have a configurable debug/verbose mode to provide further details into its operation and state.
    \end{enumerate}
\end{enumerate}

\subsection{Processor Requirements}

\begin{quote}
    ``The processor must be RISC based with a minimal but expressive instruction set.''
\end{quote}

\begin{enumerate}[label=R\arabic*]
    \item The processor's instruction set shall be reduced and minimal.
    \begin{enumerate}[label=\theenumi.\arabic*]
        \item Every instruction shall be encoded in a fixed-size.
        \item Each instruction shall be atomic.
        \item Each instruction shall be necessary.
    \end{enumerate}
    \item The processor's instruction set shall minimally be Turing-complete and expressive.
    \begin{enumerate}[label=\theenumi.\arabic*]
        \item The processor shall operate primary through registers.
        \item The ISA shall provide instructions for storing and retrieving data.
        \begin{enumerate}[label=\theenumii.\arabic*]
            \item The ISA shall provide operations to load data into registers.
            \item The ISA shall provide operations to store data into memory.
        \end{enumerate}
        \item The processor shall support both positive and negative integers, as well as floating-point numbers.
        \begin{enumerate}[label=\theenumii.\arabic*]
            \item The ISA shall support the conversion between integers and floating-point values.
        \end{enumerate}
        \item The ISA shall support arithmetic operations for both integers and floats.
        \begin{enumerate}[label=\theenumii.\arabic*]
            \item The ISA shall support a method to specify the datatype for an operation.
            \begin{enumerate}[label=\theenumiii.\arabic*]
                \item The ISA shall be able to specify a (signed?) word.
                \item The ISA shall be able to specify a (signed?) half-word.
                \item The ISA shall be able to specify an IEEE754 float and double.
            \end{enumerate}
            \item The ISA shall support addition.
            \item The ISA shall support subtraction.
            \item The ISA shall support multiplication.
            \item The ISA shall support division.
        \end{enumerate}
        \item The ISA shall support conditional branching and jumps.
        \begin{enumerate}[label=\theenumii.\arabic*]
            \item The ISA shall support the comparison between registers and values.
            \item The processor shall be able to store the result of comparisons for future use.
            \item The ISA shall support branching conditional on the result of a comparison.
        \end{enumerate}
        \item The ISA shall support a complete set of logical operations.
        \begin{enumerate}[label=\theenumii.\arabic*]
            \item The ISA shall support bitwise-not.
            \item The ISA shall support bitwise-and.
            \item The ISA shall support bitwise-or.
            \item The ISA shall support both logical left-- and right-shifting.
        \end{enumerate}
    \end{enumerate}
\end{enumerate}
\medskip

The following additional requirements extend the above requirements by adding more expressiveness to the processor:

\medskip
\begin{enumerate}[label=AR\arabic*]
    \item The ISA shall provide pseudo-instructions for common tasks.
    \item The ISA shall provide instructions for common operations which are covered but may be complex or multi-stepped to do so.
    \begin{enumerate}[label=\theenumi.\arabic*]
        \item The ISA shall provide a modulo operation.
        \item The ISA shall provide a bitwise-exclusive-or operation.
    \end{enumerate}
    \item The ISA shall support functionality to call and return from procedures.
    \begin{enumerate}[label=\theenumi.\arabic*]
        \item The processor shall support the implementation and operation of a global stack structure.
        \begin{enumerate}[label=\theenumii.\arabic*]
            \item The processor shall have a stack pointer register.
            \item The ISA shall support pushing values to the stack.
            \item The ISA shall support popping values off the stack.
        \end{enumerate}
        \item The processor shall support the concept of stack frames.
        \begin{enumerate}[label=\theenumii.\arabic*]
            \item The processor shall have a frame pointer register.
        \end{enumerate}
    \end{enumerate}
    \item The processor shall support configurable interrupts.
    \begin{enumerate}[label=\theenumi.\arabic*]
        \item The processor shall have an interrupt status register.
        \item The processor shall have an interrupt mask register.
        \item The processor shall check for the presence of an interrupt every fetch-execute cycle.
    \end{enumerate}
    \item The processor shall provide functionality syscall functionality for additional useful behaviours.
    \begin{enumerate}[label=\theenumi.\arabic*]
        \item The processor shall provide syscalls for printing data to stdout.
        \item The processor shall provide syscalls for inputting data to stdin.
        \item The processor shall provide syscalls for halting the current process.
    \end{enumerate}
\end{enumerate}

\section{The Assembler}

An intermediary between machine code for the emulator and the high-level language, the assembler's role is to take generated assembly code and output a binary.

\subsection{Assembler Requirements}

\begin{quote}
    ``The assembler must convert an assembly source to a machine-code binary executable on the emulator.''
\end{quote}

\begin{enumerate}[label=R\arabic*]
    \item The assembler shall be able to accept, read, assemble a text file, and output a binary.
    \begin{enumerate}[label=\theenumi.\arabic*]
        \item The assembler shall be able to accept a file name as an argument.
        \item The assembler shall be able to open and read a file in text mode.
        \item The assembler shall be able to write to a file in binary mode.
    \end{enumerate}
    \item The assembler shall transform a text source to a binary according to a specification.
    \begin{enumerate}[label=\theenumi.\arabic*]
        \item The assembler shall implement an assembly language specification.
        \begin{enumerate}[label=\theenumii.\arabic*]
            \item The specification shall specify the overall structure and syntax of the assembly code.
            \item The specification shall outline and describe mnemonics to implement the processor specification.
            \item The specification shall outline argument formats.
        \end{enumerate}
        \item The output assembly shall be executable with no non-user errors by the processor.
    \end{enumerate}
\end{enumerate}

\subsection{Assembly Language Requirements}

\begin{quote}
    ``The assembly language must be able to express all the processor's operations and functions, serving as a more abstract but equally powerful medium for programming the processor.''
\end{quote}

\begin{enumerate}[label=R\arabic*]
    \item The assembly shall be at least as powerful as the processor.
    \begin{enumerate}[label=\theenumi.\arabic*]
        \item The assembly shall implement all instructions stated in the processor specification.
        \item The assembly shall support access to all available memory regions of the processor.
    \end{enumerate}
    \item The assembly shall provide a layer of abstraction over raw processor binary.
    \begin{enumerate}[label=\theenumi.\arabic*]
        \item Assembly mnemonics shall be short, easy to remember, and relate to the instruction.
        \item Assembly arguments shall be explicit and use of different formats obvious to recognise.
        \item The assembly shall support labels as placeholders for addresses to ensure ease of programming.
    \end{enumerate}
\end{enumerate}

\section{The Compiler}

Arguably the entire reason for the project, a high-level language will be designed and documented.
The compiler will simply implement this specification, outputting one or more assembly files for the assembler.

\subsection{Compiler Requirements}

\begin{quote}
    ``The compiler must be able to accept one or more files, read them, and compile the contents to one or more assembly files.''
\end{quote}

\begin{enumerate}[label=R\arabic*]
    \item The compiler shall be able to accept, read, assemble a text file, and output an assembly file.
    \begin{enumerate}[label=\theenumi.\arabic*]
        \item The compiler shall be able to accept a file name as an argument.
        \item The compiler shall be able to open and read a file in text mode.
        \item The compiler shall be able to write to a file in text mode.
    \end{enumerate}
    \item The compiler shall transform a text source from the language to a text assembly source according to a specification.
    \begin{enumerate}[label=\theenumi.\arabic*]
        \item The compiler shall implement a language specification.
        \begin{enumerate}[label=\theenumii.\arabic*]
            \item The specification shall specify the overall structure and syntax of the code.
            \item The specification shall specify how to create, read, and write to variables.
            \item The specification shall outline control structures.
        \end{enumerate}
        \item The compiler shall implement a language specification.
    \end{enumerate}
\end{enumerate}

\subsection{High-Level Language Requirements}

\begin{quote}
    ``The language must be able to provide structures or methods to fully exercise the assembly language, but provide sufficient abstraction as to reveal no design intricacies or details, and be easy to read and understand.''
\end{quote}

\begin{enumerate}[label=R\arabic*]
    \item The language shall be at least as powerful as the assembly language.
    \begin{enumerate}[label=\theenumi.\arabic*]
        \item The language shall implement or provide structures for all assembly instructions.
    \end{enumerate}
    \item The language shall provide a layer of abstraction over assembly code.
    \begin{enumerate}[label=\theenumi.\arabic*]
        \item No assembly mnemonic shall be necessary to directly reference.
        \begin{enumerate}[label=\theenumii.\arabic*]
            \item The language shall provide standard infix operators to achieve operations between data.
            \item The language shall provide semantic structures to express branching operations.
        \end{enumerate}
        \item The language shall provide the ability to address registers and memory locations without explicit knowledge of the programmer.
    \end{enumerate}
\end{enumerate}

\end{document}