\section{Types}\label{sec:types}

Continuing the analogy from variables, a type is a label that tells the program what kind of data a variable holds.
Every variable and value has a type; a value without a type is meaningless, as a type gives data a meaning.

\subsection{Primitive Types}

Primitive types are a subset of those provided by the compiler representing atomic values, such as numbers.

\medskip
\begin{table}[H]
    \centering
    \begin{tabular}{|c|c|c|l|}
        \hline
        \textbf{Type} & \textbf{Alias} & \textbf{Description} \\
        \hline
        \texttt{bool} & & A truth value -- either \texttt{true} or \texttt{false} \\
        \hline
        \texttt{u8} & \texttt{byte} & Unsigned 8-bit integer \\
        \texttt{i8} & & Signed 8-bit integer \\
        \hline
        \texttt{u16} & & Unsigned 16-bit integer \\
        \texttt{i16} & & Signed 16-bit integer \\
        \hline
        \texttt{u32} & & Unsigned 32-bit integer \\
        \texttt{i32} & \texttt{int} & Signed 32-bit integer \\
        \hline
        \texttt{u64} & & Unsigned 64-bit integer \\
        \texttt{i64} & \texttt{long} & Signed 64-bit integer \\
        \hline
        \texttt{f32} & \texttt{float} & 32-bit floating-point number \\
        \texttt{f64} & \texttt{double} & 64-bit floating-point number \\
        \hline
    \end{tabular}
    \caption{Primitive Types}\label{tab:primitive-types}
\end{table}

% TODO type hierarchy

These types have the following type hierarchy:
\begin{itemize}
    \item \texttt{int8 < int16 < int32 < int64}
    \item \texttt{uint8 < uint16 < uint32 < uint64}
    \item \texttt{uint\(n\) < int\((n+k)\)}
    \item \texttt{float32 < float64}
    \item \texttt{[u]int\(n\) < float64}
    \item \texttt{[u]int\(n\) < float32} where \(n < 64\).
\end{itemize}

Note that \texttt{bool} is not related to anything.
Unlike in other language, \texttt{bool} is treated as a distinct type with abstract true/false values, rather than a synonym for integers 1/0, respectively.

\subsection{The Unit Type}

Ths unit type, denotes by empty brackets `\texttt{()}', is a special type meaning \textit{empty}.
Representing an object with no value, it is a zero-width type, meaning a symbol of this type occupies no space.
For examples of use, a unit is returned from a block which returns no value.

\begin{lstlisting}[language=CustomLang]
let a = {}; // a: ()
\end{lstlisting}

Note that a unit is neither an r-- nor an l-value, so it cannot be used in expressions.
That is, the below program errors before even trying to find a candidate for \texttt{+}.

\begin{lstlisting}[language=CustomLang]
() + 1; // error: expected rvalue, got ()
\end{lstlisting}

\subsubsection{Zero-Sized Types}

It has been mentioned that \texttt{()} is a zero-sized type, meaning it takes up no physical space.
These types exist purely for type-checking and compile-time enforcement of constraints but do not contribute to a program's memory footprint.

\begin{lstlisting}[language=CustomLang]
operator +(a: int, b: ()) {
    return ();
}
let ans = 1 + (); // = ()
\end{lstlisting}

In the above snippet, as \texttt{a} has type \texttt{()}, it does not occupy any space on the stack.
However, the assignment expression is still executed, despite knowing the result is \texttt{()} and hence discardable, to ensure any side-effects are captured.

\subsection{Pointer Types}\label{subsec:pointer-types}

The use of pointers is designed to be minimal in Edel, but they are available nonetheless.
Simply put, a pointer stores the \textit{location} (or \textit{address}) of a variable, which can then be used to later reference and access that data.
A pointer is indicated by a star `\texttt{*}' preceding the type;
that is, type \texttt{*T} is telling the compiler \textit{hey, see this location? Data of type} \texttt{T} \textit{is stored there}.

Subsection ~\ref{subsec:address-of-&-dereferencing} explains the use of the \texttt{\&} and \texttt{*} unary operators to manipulate pointers.

Pointers do not fit into the type hierarchy, and cannot be implicitly cast.
For safety, it is best not to cast pointer types unless one really knows and understands what is going on.

\subsection{Function Types}\label{subsec:function-types}

% TODO

\subsection{Casting}\label{subsec:type-casting}

Casting refers to the changing of a type, and can be split into two categories: implicit, and explicit.
Note that the destination type of either cast \textbf{cannot} be zero-sized.

Implicit casting is where a cast is done behind-the-scenes.
Type \(a\) may be implicitly cast (also known as \textit{coerced}) to another type \(b\) if and only if \(a < b\), that is, \(a\) is a subtype of \(b\).
In other words, \(a\) can be coerced into \(b\) if \(b\) is sufficient to represent \(a\) without data loss.
For example, \texttt{u8} may be coerced into \texttt{u64}, or \texttt{float} into \texttt{double}, but \textbf{not} \texttt{i32} into \texttt{i16}.
Implicit casting is done automatically in the following:

\begin{lstlisting}[language=CustomLang]
let a: double = 3.14; // float -> double
\end{lstlisting}

On the other hand, explicit casting violates these rules, and hence the cast may be \textit{explicit}.
Note that not all datatypes may be cast between each-other using explicit casting, but the subtype relationships may be largely violated.

\begin{lstlisting}[language=CustomLang]
let a: u8 = 42; // error!
let a = 42 as u8; // i32 -> u8
\end{lstlisting}
