\section{Introduction}\label{sec:introduction}

This document will contain information about the Edel language.
As an educational language, it is designed to be simple and easy to learn and operate, with the goals of being clear a readable.
Each section is written from the perspective of little programming knowledge.

\subsection{Overview}\label{subsec:overview}

As the most wide-spread paradigm, this language is imperative by nature.
Its syntax is inspired by C, but borrows features from C++ such as namespaces and operator overloading.
However, unlike these two languages, names (except variables) are \textit{forward declared}, meaning they can be used prior to their definition.
This foregoes the need for header files of explicit forward declarations as in C and C++, and allows a much freer programming experience similar to modern languages.

% TODO talk about imports

\subsection{The Compiler}\label{subsec:the-compiler}

A program consists of one file (the entry file), which is compiled by the compiler to produce an assembly output.

\medskip
\begin{lstlisting}[style=bashconsole]
$ ./compiler <input_file> [-o <output_file>] [flags]
\end{lstlisting}
\medskip

Where \texttt{[flags]} are as follows:
\begin{itemize}
    \item \texttt{--ast} - prints the parsed program's AST.
    \item \texttt{--[no-]function-placeholder} - changes behaviour of a function declaration; see subsection ~\ref{subsec:function-declaration}.
    \item \texttt{--[no-]indentation} - adds/removes indentation from generated assembly.
    \item \texttt{--[no-]lint} - enables/disables linting messages.
    \item \texttt{--lint-level \(n\)} - sets the linting error reporting level.
    \begin{enumerate} \setcounter{enumi}{-1}
        \item Note.
        \item Warning.
        \item Error.
    \end{enumerate}
\end{itemize}
Note that the output file is optional.

\subsection{``Hello, World''}\label{subsec:hello-world}

As is customary, a quick ``hello world'' program written in Edel.

\begin{lstlisting}[language=CustomLang]
func main() {
    print("Hello, world\n");
}
\end{lstlisting}

\subsection{Linting}

As this compiler focuses on ease-of-use, it does not contain many linting options or messages.
However, some messages such as empty blocks/namespaces have been included in hopes to alert programmers of unintentional happenings.