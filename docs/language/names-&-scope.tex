\section{Names \& Scope}\label{sec:names-&-scope}

A \textit{name} in Edel refers to an identifier, which includes functions, variables, and namespaces.
All names have a \textit{scope}, which is the area in which that name exists/may be referenced.
That is, outside its scope, a symbol does not exist.

The outermost scope is known as the \textit{global} scope.
Names defined here will be available anywhere in the current file.

The counterpart is the \textit{local} scope, which refers to the latest scope.
New scopes are created by using code blocks, for example, in while loops and functions.

\subsection{Variables}\label{subsec:variables}

Using the common analogy, a variable is like a labeled box which stored data.
Each box has a \textit{type} which tells us what it stores.
For example, ``int 3''.
More technically, a variable is a location in memory where data can be stored, with its type determining the size of the variable and how its value can be used.

Variables are defined using the \texttt{let} keyword, with the variables type following a colon.
For example, to define our ``int 3'' from earlier,
\begin{lstlisting}[language=CustomLang]
let foo: int = 3;
\end{lstlisting}

If required, multiple variables can be defined in the same \texttt{let} statement.
\begin{lstlisting}[language=CustomLang]
let foo: int = 3, bar: float = 3.14;
\end{lstlisting}

If the type is omitted, it is deduced from the assigned value.
Hence, this can only be done if the variable is assigned.
\begin{lstlisting}[language=CustomLang]
let foo = 3, bar = 3.14;
\end{lstlisting}
(In this case, both variables adopt the types as previously written.)

\subsubsection{Shadowing}

Edel allows names to be \textit{shadowed}, meaning a variable may re-use a variable's name.

In non-local shadowing, the previous definition becomes visible again after the current scope is exited.
\begin{lstlisting}[language=CustomLang]
let foo = 1;
{
    let foo = 2;
    // foo = 2
}
// foo = 1
\end{lstlisting}

Edel also allows local shadowing.
In this case, once shadowed, the previous definition is essentially lost and cannot be referenced.
\begin{lstlisting}[language=CustomLang]
let foo = 1;
// foo = 1
let foo = 2;
// foo = 2
\end{lstlisting}

\subsubsection{Constants}

Constants are values that, once defined, cannot be changed.
Constants may be defined using the \texttt{const} keyword in place of \texttt{let}.

\begin{lstlisting}[language=CustomLang]
const pi = 3.14159;
\end{lstlisting}

In the case of multiple definitions, \textit{all} symbols will be defined as const.
