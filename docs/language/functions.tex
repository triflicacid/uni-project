\section{Functions}\label{sec:functions}

Functions are labelled blocks of code which may be called by the programmer.
A function may take some input values, known as \textit{arguments}, and may optionally return a value.

An example of a function which takes two integers and returns an integer:
\begin{lstlisting}[language=CustomLang]
func add(a: int, b: int) -> int {
    return a + b;
}
\end{lstlisting}

A function is defined using the \texttt{func} keyword, followed by a name.
Parameters in the form \texttt{name: type} are listed, comma-separated, between brackets.
Note that if no arguments are required, the brackets may be omitted altogether.
The function's return type is listed after an arrow; if no return type is needed, this may also be omitted.
For example,
\begin{lstlisting}[language=CustomLang]
func foo { // signature food() -> ()
    // ...
}

func bar(a: float) { // signature foo(float) -> ()
    // ...
}
\end{lstlisting}

\subsection{Calling Functions}

Functions are called by following the function name with brackets `\texttt{()}' (this is the operator name).
Any arguments to be supplied to the function are comma-separated inside these brackets, known as an \textit{argument list}.
A function with a matching signature is then found (see subsection ~\ref{subsec:function-overloading} on function overloading), and its return value produced.

\begin{lstlisting}[language=CustomLang]
func add(a: int, b: int) -> int {
    return a + b;
}
let result = add(1, 4); // result: int = 5
\end{lstlisting}

Note that \texttt{()} is an operator like any other.
Therefore, if the subject is not a function type, an overload for \texttt{operator()} is searched for.
The signature will be the subject type followed by each argument's type.
For example,

\begin{lstlisting}[language=CustomLang]
5(3.14, true); // operator()(i32, f32, bool)
\end{lstlisting}

\subsection{Returning a Value}

The \texttt{return} keyword is used to exit/return from the current function.
Optionally, the keyword may be followed by an expression, in which case that value will be returned.
The type of the returned expression must match, or be a subtype of, the function's return type.
Furthermore, if the function returns a non-unit type, a return statement must be present.

\begin{lstlisting}[language=CustomLang]
func foo {
    return 42; // error: cannot convert i32 to ()
}

func bar -> int {
} // error: missing return statement
\end{lstlisting}

Note that \textit{all} code paths should return, even if you, the programmer, knows that a path is never taken.
% TODO update if constant optimisation added
That is because the compiler cannot currently recognise, for example, always-falsy conditions.

\begin{lstlisting}[language=CustomLang]
func foo -> int {
    if false {
    } else {
        return 42;
    }
} // error: missing return statement
\end{lstlisting}

\subsubsection{Unreachable Code}

As \texttt{return} exits the function, any code after said statement is \textit{unreachable}.
This code will never be executed, and a warning will be reported.

\begin{lstlisting}[language=CustomLang]
func foo {
    return;
    a; // warning: unreachable code detected
}
\end{lstlisting}

The above code would usually result in an error, as symbol `\texttt{a}' is not defined.
However, as the code is not reached, no error is raised.

\subsection{Function Declaration}\label{subsec:function-declaration}

A function declaration is one which has no body.
For example,

\begin{lstlisting}[language=CustomLang]
func add(a: int, b: int) -> int;
\end{lstlisting}

As functions in Edel do not require forward-declaration, what does this do?
Depending on the compiler flag \texttt{--function-placeholder}, it does one of two things.

\paragraph*{Enforce Existence}

The default, \texttt{--no-function-placeholder}, a function declaration is used to inform the programmer about the existence of a function.
One common use is in auto-generated standard library files, where the function exists but is implemented internally.
If the function does not exist, an error will be raised.

\begin{lstlisting}[language=CustomLang]
func imaginary_function(x: int) -> bool; // error: function was declared but does not exist
\end{lstlisting}

\paragraph*{Generate Placeholder}

In this instance, if a function is declared but does not exist, the function will be created to return zero, `\texttt{()}', false, or null, depending on the return type.
This is useful when a function is yet to be implemented.

For example, the following are identical.
\begin{lstlisting}[language=CustomLang]
func imaginary_function(x: int) -> bool;
// ==
func imaginary_function(x: int) -> bool {
    return false;
}
\end{lstlisting}


Note that as this generates a definition, it cannot be re-defined.

\subsection{Function Overloading}\label{subsec:function-overloading}

A function in Edel is not only identified by its name, but also its \textit{signature}.
The signature of a function is a combination of its name and parameter types.
In an Edel program, a function signature must be unique, or the result would be an error.
This means that multiple functions with the same name can exist.
This is known as function overloading, and a variant of a function with the same name but different parameters is called an \textit{overload}.

\begin{lstlisting}[language=CustomLang]
func add(a: int, b: int) -> int {
    return a + b;
}
func add(a: float, b: float) -> float {
    return a + b;
}
\end{lstlisting}

In this case, a reference to \texttt{add} no longer just refers to one symbol.
Instead, it becomes what is known as an \textit{overload set}, and more information is required for a resolution.

\begin{lstlisting}[language=CustomLang]
add; // error: unable to resolve symbol `add'
\end{lstlisting}

\subsubsection{Overload Resolution}

This is the process by which an overload is selected.
Firstly, a list of candidates is generated; this would be all functions with a matching name.
A candidate is considered a match if the number of arguments match and the type of each parameter matches.
Due to subtyping rules, the latter check may lead to the generating of multiple `matching' candidates.
In such a case, the candidate with the highest number of identical types wins.

\begin{lstlisting}[language=CustomLang]
func add(a: int, b: int) -> int;
func add(a: long, b: long) -> long;

add(1, 2); // signature is add(int, int)
\end{lstlisting}

In the above snippet, both candidates match the callee.
However, the former wins as \texttt{int == int} while \texttt{int < long}.

If there are still ties at this point, an error is raised as a candidate cannot be decided.
In this case, it is best to explicitly cast one or more arguments to reach the desired candidate.

\begin{lstlisting}[language=CustomLang]
func add(a: int, b: long) -> long;
func add(a: long, b: int) -> long;

add(1, 2); // error: unable to resolve symbol `add'
\end{lstlisting}
