\section{Basic Syntax}\label{sec:basic-syntax}

Inspired by C, Edel is not whitespace sensitive, with scopes marked by braces \texttt{\{\}}.
Semicolons are not requires after every line, namely closing braces, but are required after expressions.

Comments come in two forms: single-- and multi-line.
Single-line comments \texttt{// ...} comment out the remainder of the current line;
multi-line comments \texttt{/* ... */} comment out anything they enclose.

\subsection{Expressions \& Statements}\label{subsec:expressions-&-statements}

Generally, a statement is some code which does not produce a value, whereas an expression does.
In Edel, \textit{everything} is an expression, with statements, by default, returning the unit type `\texttt{()}'.

Expressions are expected to be terminated by a semicolon.
The exception to this is the final expression in a block; if omitted, the block `returns' the result of this expression.

\begin{lstlisting}[language=CustomLang]
let a = { 1 + 2; }; // a = ()
let b = { 1 + 2 }; // b = 3
let c = { let z = 2 * b; z } + 2; // c = 8
\end{lstlisting}

Note that only the last expression in a block may have their semicolon emitted.
By always terminating expressions, the block will act like a statement.

\subsubsection{Value Categories}

Each expression, be it an operator with operands, literals, or variable names etc., is characterised by a \textit{type} and a \textit{category}.
Types, covered in section ~\ref{sec:types}, give meaning to data and what it means in context, whereas the category tells dictates \textit{how} a value can be used.

Edel has two categories, be them l-- and r-values.
The former is a value which refers to a memory location, such as variables, while the latter has no identifiable location in memory.
The \textit{l} and \textit{r} stand for \textit{left} and \textit{right} because the respective values may appear on the left-- or right-hand side of an assignment operator.
That is,

\begin{lstlisting}[language=CustomLang]
a = 1; // a (lvalue) = 1 (rvalue)
\end{lstlisting}

As an rvalue has no identifiable location, it cannot be assigned to, nor can it be access via the dot `\texttt{.}' operator.
Instances of misuse of an l-- or rvalue results in an error ``\texttt{expected l/rvalue, got type}''.

\begin{lstlisting}[language=CustomLang]
1 = 2; // error: expected lvalue, got i32
true.a; // error: expected lvalue, got bool

namespace maths {}
maths + 1; // error: expected rvalue, got namespace
\end{lstlisting}

\subsection{Code Blocks}\label{subsec:code-blocks}

As a whitespace-ignorant language, Edel uses blocks to denote structure and scope.
A block is introduced by braces \texttt{\{\}} and contains a section of code.
As previously mentioned, a block may be an expression or a statement, hence is permitted anywhere they would be expected.

Blocks in Edel introduce a new lexical scope (see section ~\ref{sec:names-&-scope} for scope resolution).
