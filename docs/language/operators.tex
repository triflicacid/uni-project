\section{Operators}\label{sec:operators}

Operators are special symbols which combine one or two expressions, known as unary and binary operators, respectively.
In Edel, operators are generic, and any string containing the following characters is parsed as an operator:

\begin{center}
    \texttt{!\#\$\%\&*+./<=>?@{\string^}|-~}
\end{center}

As operators are generic (rather than a pre-defined set of symbols), certain information is determined by its use.
Namely, an operator's \textit{arity} is determined by its placement: unary if it is seen preceding a term, binary otherwise.

An operator's \textit{associativity} determines the direction in which a chain of said operators are evaluated.
As an example, let \(\sim\) be a binary operator in the expression \(a \sim b \sim c\).
\begin{itemize}
    \item Left-associative (ltr, \(\longrightarrow\)): equivalent to \((a \sim b) \sim c\).
    \item Right-associative (rtl, \(\longleftarrow\)): equivalent to \(a \sim (b \sim c)\).
\end{itemize}

Finally, operators have a \textit{precedence}.
This is how tightly an operator binds, and is useful to avoid having to explicitly include brackets everywhere.
For example, \(a + b * c \equiv a + (b * c)\) but \textbf{not} \(a + b) * c\) as \texttt{*} binds tighter than \texttt{+}.

The table below displays a list of built-in operators alongside their associativity and precedence.
The associativity and precedence of a generic (non built-in) operator is also shown.

\begin{table}[h]
    \centering
    \caption{Operator Precedence and Associativity}
    \begin{tabular}{|l|l|c|c|}
        \hline
        \textbf{Operator} & \textbf{Description} & \textbf{Associativity} & \textbf{Overloadable} \\
        \hline
        \texttt{a.b} & Member access & \(\longrightarrow\) & No \\
        \hline
        \texttt{a()} & Function call & \multirow{2}{*}{\(\longrightarrow\)} & \multirow{2}{*}{Yes} \\
        \texttt{a[]} & Subscript & & \\
        \hline
        \texttt{(T) a} & Primitive cast & \multirow{5}{*}{\(\longleftarrow\)} & \multirow{5}{*}{No} \\
        \texttt{\&a} & Address-of & & \\
        \texttt{*a} & Dereference & & \\
        \texttt{registerof a} & Register lookup & & \\
        \texttt{sizeof a} & Get size & & \\
        \hline
        \texttt{-a} & Negation & \multirow{4}{*}{\(\longleftarrow\)} & \multirow{4}{*}{Yes} \\
        \texttt{~a} & Bitwise NOT & & \\
        \texttt{!a} & Boolean NOT & & \\
        \(\langle op \rangle\)\texttt{a} & \textit{Generic} & & \\
        \hline
        \texttt{a as T} & Cast & \(\longleftarrow\) & No \\
        \hline
        \texttt{a * b} & Multiplication & \multirow{3}{*}{\(\longrightarrow\)} & \multirow{3}{*}{Yes} \\
        \texttt{a / b} & Division & & \\
        \texttt{a \% b} & Modulo (remainder) & & \\
        \hline
        \texttt{a + b} & Addition & \multirow{2}{*}{\(\longrightarrow\)} & \multirow{2}{*}{Yes} \\
        \texttt{a - b} & Subtraction & & \\
        \hline
        \texttt{a << b} & Bitwise left shift & \multirow{2}{*}{\(\longrightarrow\)} & \multirow{2}{*}{Yes} \\
        \texttt{a >> b} & Bitwise left shift & & \\
        \hline
        \texttt{a == b} & \multirow{6}{*}{Relational operators} & \multirow{6}{*}{\(\longrightarrow\)} & \multirow{6}{*}{Yes} \\
        \texttt{a != b} & & & \\
        \texttt{a < b} & & & \\
        \texttt{a <= b} & & & \\
        \texttt{a > b} & & & \\
        \texttt{a >= b} & & & \\
        \hline
        \texttt{a \& b} & Bitwise AND & \multirow{1}{*}{\(\longrightarrow\)} & Yes \\
        \hline
        \texttt{a \string^ b} & Bitwise XOR & \multirow{1}{*}{\(\longrightarrow\)} & Yes \\
        \hline
        \texttt{a | b} & Bitwise OR & \multirow{1}{*}{\(\longrightarrow\)} & Yes \\
        \hline
        \texttt{a \(\langle op \rangle\) b} & \textit{Generic} & \multirow{1}{*}{\(\longrightarrow\)} & Yes \\
        \hline
        \texttt{a = b} & Assignment & \multirow{1}{*}{\(\longleftarrow\)} & No \\
        \hline
    \end{tabular}\label{tab:builtin-operators}
\end{table}

\subsection{Casting}\label{subsec:op-casting}

The concept of casting is explained in subsection ~\ref{subsec:type-casting}.
Edel offers two casting operators: \texttt{(T) a} and \texttt{a as T}.

The former is a primitive cast, meaning it only supports casting to primitive, atomic types.
This cast is direct and is ignorant of the type of \texttt{a} and whether the cast is meaningful.

The latter, using the \texttt{as} keyword, is a straightforward explicit cast.
There are no restrictions on \texttt{T} but, as explained in subsection ~\ref{subsec:type-casting}, this cast still must obey some rules and will fail otherwise.

\begin{lstlisting}[language=CustomLang]
let a = (int) 3.14; // ok
let b = (*int) 0; // syntax error
let c = 0 as *int; // ok
\end{lstlisting}

Also note the difference in precedence: the \texttt{as} cast is slightly looser than other unary operators.

\begin{lstlisting}[language=CustomLang]
!1 as bool; // same as `(!1) as bool' NOT `!(1 as bool)'
\end{lstlisting}

\subsection{Address-Of \& Dereferencing}\label{subsec:address-of-&-dereferencing}

Both operations involve pointers, with the former generating pointers, and the latter `consuming' them.

The address-of operator, \texttt{\&}, calculates the memory address of a symbol.
That is, if \texttt{a: T}, then \texttt{\&a} is of type \texttt{*T}.

Dereferencing, then, is address-of's twin; it retrieved the value stored at a pointer.
That is, if \texttt{p: *T}, then \texttt{*a} is of type \texttt{T}.

\begin{lstlisting}[language=CustomLang]
let a: int = 5;
let p: *int = &a;
let b: int = *p; // then a == b
\end{lstlisting}

The dereference operator returns an lvalue, so it may be assigned to.
In this case, the memory pointed to by the pointer will be written to.
Continuing the example from above,

\begin{lstlisting}[language=CustomLang]
*p = 10; // a = 10
\end{lstlisting}

However, \texttt{b} will still be 5 as it was assigned to \texttt{*p} previously and is no longer tied to the same address.

\subsection{Register Lookup}

This is a lower-level operator meant for exploration of the processor's registers.

Expecting a symbol argument, the \texttt{registerof} operator returns the index (offset) of where the given symbol is located.
This allows one to gain insight into the register allocation algorithm.

For example, given the basic program below, one would expect the following.

\begin{lstlisting}[language=CustomLang]
let a = 5;
registerof a; // = 11, corresponding to $r1
\end{lstlisting}

If the symbol has not been loaded into registers (i.e., it has not been used at all/recently), the operator returns \(-1\).

\begin{lstlisting}[language=CustomLang]
let a: int;
registerof a; // = -1
\end{lstlisting}

\subsection{Sizeof}

The \texttt{sizeof} operator returns the size, in bytes (as a \texttt{u64}), of its argument.
Since the argument to \texttt{sizeof} can be either a type or an expression, the returned value corresponds to the size of the provided type or, if given an expression, the size of its type.

\begin{lstlisting}[language=CustomLang]
sizeof(int); // 4
sizeof(42); // 4
\end{lstlisting}

(The brackets are not required as \texttt{sizeof} is an operator and are shown for clarity only.)

\subsection{Operator Overloading}

Much like functions, operators may be overridden, but this also allows new, custom operators to be defined.
An operator overload is defined much like a function, except starting with the \texttt{operator} keyword, followed by the operator symbol.
For example,

\begin{lstlisting}[language=CustomLang]
operator $(x: int) -> int {
    return x * 100;
}
\end{lstlisting}

defines a new unary overload for the operator \texttt{\$} when applied to an integer.
Similarly, one may define a binary overload.

\begin{lstlisting}[language=CustomLang]
operator $(a: int, b: int) -> int {
    return $a + $b;
}
\end{lstlisting}

Given these definitions, one could then do the following.

\begin{lstlisting}[language=CustomLang]
let a = $5; // = 5 * 100 = 500
let b = 1 $ 2; // = 1 * 100 + 2 * 100 = 300
\end{lstlisting}

It is perfectly permissible to overload built-in operators.
Just note that these operators have a non-standard associativity and precedence.

\begin{lstlisting}[language=CustomLang]
operator +(a: int, b: float) -> float {
    return (float) a + b;
}
let ans = 1 + 3.14; // = 4.14
\end{lstlisting}

Like functions, an existing operator overload cannot be shadowed.
For example, the following would fail.

\begin{lstlisting}[language=CustomLang]
operator +(a: int, b: int) -> int { // error: operator+(i32, i32) already exists
    // ...
}
\end{lstlisting}
