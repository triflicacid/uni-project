\section{Control Flow}\label{sec:control-flow}

Control flow refers to code which may alter the otherwise-linear flow of code.
Without control flow, all programs would be deterministic and extremely limiting.

\subsection{If Statements}

An if statement is a basic conditional: if the condition, or \textit{guard}, is true, it executes one section of code, otherwise executed another.
The syntax of an if statement is as follows.

\begin{lstlisting}[language=CustomLang]
if condition {
    // truthy code
} else {
    // falsy code
}
\end{lstlisting}

With \texttt{condition} being a Boolean value.
Note that the \texttt{else} branch is optional and may be omitted if not required.

\subsubsection{If Expressions}

Much like blocks, if statements may also be used as an expression by omitting the last semicolon in the branches.
When used this way, both branches must return values of the \textit{exact} same type.
As a result, an if expression must always include an \texttt{else} branch.

\begin{lstlisting}[language=CustomLang]
let x = if a > 0 { 1 } else { -1 };
\end{lstlisting}

Here, both branches return an \texttt{int}, so the expression is valid.

However, if a branch always returns, its evaluated type is ignored since execution will never reach the rest of the expression.

\begin{lstlisting}[language=CustomLang]
let x = if a > 0 {
    1
} else {
    return false;
}
\end{lstlisting}

In this case, the \texttt{else} branch does not produce a value -- it causes the function to return instead.
Since the \texttt{else} branch never completes normally, its return type does not affect the type-checking of the if expression, and the code remains valid.

\subsubsection{Multiple Branches}

The effect of multiple conditional branches may be emulated by stacking if-else statements and removing the braces around all but the final \texttt{else} branch.

\begin{lstlisting}[language=CustomLang]
if C1 {
    // ...
} else if C2 {
    // ...
} else {
    // ...
}
\end{lstlisting}

Note that if both \texttt{C1} and \texttt{C2} are true, as the former is tested first, only its block will be executed before the statement is exited;
\texttt{C2} and its block will never be evaluated.

\subsection{While Loops}
